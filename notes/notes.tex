\documentclass[10pt, onecolumn, aps, prb, superscriptaddress, floatfix, showpacs, notitlepage]{revtex4-1}
\pdfoutput=1

\usepackage[utf8]{inputenc}
\usepackage[T1]{fontenc}
\usepackage{lmodern}
\usepackage{graphicx}
\usepackage{amsmath}
\usepackage{amssymb}
% \usepackage{wasysym}
\usepackage{bm}
\usepackage{xcolor}
\usepackage[colorlinks, citecolor={blue!50!black}, urlcolor={blue!50!black}]{hyperref}
\usepackage{bookmark}
\usepackage{tabularx}
\usepackage{mathtools}
\usepackage{microtype}
\usepackage[load=physical,load=abbr]{siunitx}

\DeclareMathOperator{\sgn}{sgn}
\DeclareMathOperator{\im}{Im}
\DeclareMathOperator{\re}{Re}

\newcommand{\ev}[1]{\langle#1\rangle}
\newcommand{\bra}[1]{\langle #1|}
\newcommand{\ket}[1]{|#1\rangle}
\newcommand{\bracket}[2]{\langle #1|#2\rangle}

\graphicspath{{figures/}}

\begin{document}

\author{nice people ;-)}
\affiliation
{Delft University of Technology}

\title{kpm expansion of second order perturbation matrix elements}

% \begin{abstract}
%     Abstract
% \end{abstract}
\maketitle

\section{Quasi-Degenerate Perturbation theory: standard approach}

\subsection{Problem definition}

We start by separating initial Hamiltonian into unperturbed part $H^0$ and perturbation $H^{\prime}\,$:
\begin{equation}
H = H^{0} + H^{\prime} \,.
\end{equation}

We assume that eingestates and energies of $H^{0}$ are known:
\begin{equation}
H^0 \ket{\psi_n} = \epsilon_n \ket{\psi_n} \,.
\end{equation}

We now split states $\ket{\psi_n}$ into two groups, $A$ and $B$.
We are interested in states from group $A$ whereas effect of states $B$ we want to include via perturbation theory.
We assume that these two group of states are clearly separated in energies.


\subsection{L\"owdin partitioning}
L\"owdin partitioning, see Winkler's book,~\cite{winkler} provide us with the extension of standard perturbation theory to the case of quasi-degenerate states:
\begin{equation}
\tilde{H} = e^{-s} H \, e^{s}\,,
\end{equation}
with the effective Hamiltonian given as sum of successive order of perturbation:
\begin{equation}
\tilde{H} = \tilde{H}^{(0)} + \tilde{H}^{(1)} + \tilde{H}^{(2)} + \ldots\,.
\end{equation}
For all the math, again, see Winkler's book or Rafal's thesis.
The explicit solution, for first two orders, is:
\begin{subequations}
\label{lowdin-explicit}
\begin{align}
    \tilde{H}^{(0)}_{mn} &= H^0_{mn} \,,\\
    \tilde{H}^{(1)}_{mn} &= H^1_{mn} \,,\\
    \tilde{H}^{(2)}_{mn} &= \frac{1}{2} \sum_{l\in B}
    H^{2}_{m l} H^{2}_{l n}
    \left(\frac{1}{E_m - E_l} + \frac{1}{E_n - E_l}\right)\,,
\end{align}
\end{subequations}
where $m, n$ indices go over states from group $A$, $H^{1}$ corresponds to block-diagonal part of the perturbation and $H^{2}$ is non-block-diagonal part of the perturbation.
The actual separation of $H^{\prime}$ as $H^{1}+H^{2}$ is not important here as proper blocks are taken into account automatically through correct indexing in Eq.~\eqref{lowdin-explicit}.



\subsection{Perturbation as polynomial in free parameters}
From practical point of view, we usually treat $H^{\prime}$ as polynomial in some free parameters $\alpha$
\begin{equation}
H^{\prime} = \sum_\alpha \lambda_\alpha H^{\prime}_\alpha\,,
\end{equation}
where $\lambda_\alpha$ can be in example $k_x\,$, $k_x^2\,$, $B_z\,$, etc.
Taking that into account and calculating matrix elements in Eq.~\eqref{lowdin-explicit} explicitly we obtain:
\begin{subequations}
\label{lowdin-explicit}
\begin{align}
    \tilde{H}^{(0)}_{mn} &= \epsilon_m \delta_{m,n} \,,\\
    \tilde{H}^{(1)}_{mn} &= \sum_{\alpha} \lambda_\alpha \bra{\psi_m} H^{\prime}_\alpha \ket{\psi_n} \,,\\
    \tilde{H}^{(2)}_{mn} &= \frac{1}{2} \sum_{\alpha,\beta} \lambda_\alpha\lambda_\beta\sum_{l\in B}
    \left(\frac{\bra{\psi_m}H^{\prime}_{\alpha}\ket{\psi_l}\bra{\psi_l}H^{\prime}_{\beta}\ket{\psi_n}}{E_m - E_l} + \frac{\bra{\psi_m}H^{\prime}_{\alpha}\ket{\psi_l}\bra{\psi_l}H^{\prime}_{\beta}\ket{\psi_n}}{E_n - E_l}\right)\,,
\end{align}
\end{subequations}



\section{kpm expansion}

\subsection{Main idea}

The second order contribution to the effective Hamiltonian can be expressed as
\begin{align}
\tilde{H}^{(2)}_{i,j} &= \frac{1}{2}
\bra{\psi_i} H^{\prime}
\left[
\sum_\mu
\frac{\ket{\psi_\mu} \bra{\psi_\mu}}{\epsilon_j-\epsilon_\mu}
\right]
H^{\prime} \ket{\psi_j}
+
\frac{1}{2}
\bra{\psi_i} H^{\prime}
\left[
\sum_\mu
\frac{\ket{\psi_\mu} \bra{\psi_\mu}}{\epsilon_i-\epsilon_\mu}
\right]
H^{\prime} \ket{\psi_j}
\\
&= \frac{1}{2}
\bra{\psi_i} H^{\prime}
\left[P_B
\frac{1}{\epsilon_j-H^0}
P_B\right]
H^{\prime} \ket{\psi_j}
+
\frac{1}{2}
\bra{\psi_i} H^{\prime}
\left[P_B
\frac{1}{\epsilon_i-H^0}
P_B\right]
H^{\prime} \ket{\psi_j},
\end{align}

where $P_B$ is a projector over the space $B$ defined by $\{\psi_\mu\}_\mu$,
and we see the action of a Green's function defined as

$$
G(\epsilon, H^0) = \frac{1}{\epsilon-H^0}.
$$

This function can be expanded with the KPM, with the vectors of the expansion
being

$$
\ket{\phi_i} = P_B H^{\prime}\ket{\psi_i},
$$


such that


\begin{align}
\tilde{H}^{(2)}_{i,j} &= \frac{1}{2}
\bra{\phi_i} G(\epsilon_j, H^0) \ket{\phi_j}
+
\frac{1}{2}
\bra{\phi_i} G(\epsilon_i, H^0) \ket{\phi_j}
\end{align}

This calculation can be simplified by collecting the vectors to matrices, define $\Phi$ as the matrix consisting of columns that are $\ket{\phi_i}$ and $\Xi$ whose columns are $G(\epsilon_i, H^0) \ket{\phi_i}$. With this:
\begin{equation}
\tilde{H}^{(2)} = \frac{1}{2} (\Phi^\dag \Xi + \text{h.c.})
\end{equation}

\subsection{Multiple expansion parameters}

It is common that there are multiple small parameters, i.e. the perturbed Hamiltonian is
\[H = H^0 + \sum_{\alpha} \lambda_{\alpha} H^{\prime}_{\alpha}\]
and we seek the second order effective Hamiltonian as
\[\tilde{H}_{eff} = \tilde{H}^{(0)} +  \sum_{\alpha} \lambda_{\alpha} \tilde{H}^{(1)}_{\alpha} +  \sum_{\alpha\beta} \lambda_{\alpha}\lambda_{\beta}\tilde{H}^{(2)}_{\alpha\beta}.\]
Similar as before, we introduce $\Phi_{\alpha}$ as the collection of column vectors $P_B H^{\prime}_{\alpha}\ket{\psi_i}$ and $\Xi_{\alpha}$ as $G(\epsilon_i, H^0) P_B H^{\prime}_{\alpha}\ket{\psi_i}$. With this
\[\tilde{H}^2_{\alpha\beta} = \frac{1}{2} (\Phi_{\alpha}^\dag \Xi_{\beta} + \Xi_{\alpha}^\dag\Phi_{\beta})\]
which is also valid if the expansion coefficients do not commute.

\subsection{Green's function expansion with KPM}

A function of a parameter and a Hamiltonian is expressed as

$$
f(\epsilon, H) = \sum_k f(\epsilon, E_k) \ket{\psi_k}\bra{\psi_k},
$$

where the funcion $f$ is expanded in KPM as

\begin{align}
f(\epsilon, E_k) &= \frac{2}{\pi}\sum_m c_m(\epsilon)T_m(E_k)\\
c_m(\epsilon) &=\frac{1}{1+\delta_{m,0}}
\int_{-1}^1 \frac{f(\epsilon, E_k) T_m(E_k)}{\sqrt{1-E_k^2}}\mathrm{d}E_k.
\end{align}

To simplify the calculations we can instead expand

\begin{align}
\sqrt{1-E_k^2}f(\epsilon, E_k) &= \frac{2}{\pi}\sum_m c_m(\epsilon)T_m(E_k)\\
c_m(\epsilon) &=\frac{1}{1+\delta_{m,0}}
\int_{-1}^1 {f(\epsilon, E_k) T_m(E_k)}\mathrm{d}E_k.
\end{align}


The solution is expanded elsewhere\cite{Garcia2014} for the rescaled
Hamiltonian $\tilde{H}=(H-bI)/a$, and rescaled energies
$\tilde{\epsilon}=(\epsilon-b)/a$.

\begin{align}
f(\epsilon, H)^\pm &= \frac{1}{\epsilon-H \pm i0}\\
f(\epsilon, H)^\pm &= \frac{1}{\tilde{\epsilon}a+b-\tilde{H}a-b \pm i0}\\
f(\epsilon, H)^\pm &= a^{-1}\frac{1}{\tilde{\epsilon}-\tilde{H} \pm i0}\\
f(\epsilon, H)^\pm_M &= \mp \frac{2i}{a\sqrt{1-\epsilon^2}}
\sum_{m=0}^{M}\frac{g_m}{1+\delta_{m,0}}
\exp(\pm i\,m\,\mathrm{arccos}(\epsilon)) T_m(H)
\end{align}


% In our case, we can solve the last expression as
%
% \begin{align}
% f(\epsilon, E) &= \frac{1}{\epsilon-E}\\
% T_m(E) &= \cos (m\,\mathrm{arcos}(E))\\
% \phi &= \mathrm{arcos}(E)\\
% \Rightarrow
% {\mathrm{d}E}
% &= {\mathrm{d}\cos(\phi)} = -{\sin(\phi)} \mathrm{d}\phi\\
% &\rightarrow\\
% c_m(\epsilon) &=
% \frac{1}{1+\delta_{m,0}}
% \int_0^{\pi}
% {f(\epsilon, \cos(\phi)) \cos (m\,\phi)} {\sin(\phi)}
% \mathrm{d}\phi\\
% c_m(\epsilon) &=
% \frac{1}{1+\delta_{m,0}}
% \int_0^{\pi}
% \frac{\cos (m\,\phi)\sin(\phi)}{\epsilon-\cos(\phi)}
% \mathrm{d}\phi\\
% \theta &= \epsilon - \cos(\phi)\\
% \Rightarrow \mathrm{d}\theta &= -\mathrm{d}\cos(\phi)
% \\
% &= \sin(\phi)\mathrm{d}\phi


% \section{Introduction}

% \subsection{}

% \begin{figure}[!h]
% \centering
% \includegraphics[width=\linewidth]{name}
% \caption{}
% \label{fig1}
% \end{figure}

\begin{thebibliography}{2}
\bibitem{Garcia2014}Real-space calculation of the conductivity tensor for disordered topological matter
Jose H. Garcia, Lucian Covaci, Tatiana G. Rappoport. Phys. Rev. Lett. 114, 116602 (2015)
\end{thebibliography}


\end{document}
