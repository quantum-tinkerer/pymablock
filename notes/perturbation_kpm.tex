\documentclass[10pt, onecolumn, aps, prb, superscriptaddress, floatfix, showpacs, notitlepage]{revtex4-1}
\pdfoutput=1

\usepackage[utf8]{inputenc}
\usepackage[T1]{fontenc}
\usepackage{lmodern}
\usepackage{graphicx}
\usepackage{amsmath}
\usepackage{amssymb}
% \usepackage{wasysym}
\usepackage{bm}
\usepackage{xcolor}
\usepackage[colorlinks, citecolor={blue!50!black}, urlcolor={blue!50!black}]{hyperref}
\usepackage{bookmark}
\usepackage{tabularx}
\usepackage{mathtools}
\usepackage{microtype}
\usepackage[load=physical,load=abbr]{siunitx}

\DeclareMathOperator{\sgn}{sgn}
\DeclareMathOperator{\im}{Im}
\DeclareMathOperator{\re}{Re}

\newcommand{\ev}[1]{\langle#1\rangle}
\newcommand{\bra}[1]{\langle #1|}
\newcommand{\ket}[1]{|#1\rangle}
\newcommand{\bracket}[2]{\langle #1|#2\rangle}

\graphicspath{{figures/}}

\begin{document}

\author{PM Perez-Piskunow}
\affiliation
{Delft University of Technology}

\title{kpm expansion of second order perturbation matrix elements}

% \begin{abstract}
%     Abstract
% \end{abstract}
\maketitle

\textbf{$\mathbf{H^0}$ initial Hamiltonian}

We start with an initial system $H^0$, and solve the initial
eigenvalues problem

$$
H^0 \ket{\psi_n} = \epsilon_n \ket{\psi_n}.
$$

We express the initial Hamiltonian in it's diagonal basis
$$
\tilde{H^0}_{i,j} = \delta_{i,j}\epsilon_i.
$$

\textbf{Perturbation}

We add a perturbation $\lambda H^1$, where $\lambda$ is a real parameter in $[0,1]$.

and we want to know what is the effect of this perturbation on the spectrum
of $H = H^0 + \lambda H^1$.

From non-degenerate perturbation theory, we can get the correction to the
eigenstates and eigenvalues of $H^0$ by expanding in a series of powers in
$\lambda$. Since degeneracies can occur, we must find an effective Hamiltonian
that couples the degenerate states.


\textbf{First order}

The first thing we can do is to express $H^1$ in the basis diagonalises $H^0$,

$$
\tilde{H^1}_{i,j} = \bra{\psi_i} H^1 \ket{\psi_j}.
$$

The diagonal elements $\tilde{H^1}_{i,i}$ are the first order corrections
to the energy of the perturbed system, and the effective Hamiltonian up to
first order is

$$
H_{eff}^1 = \tilde{H^0} + \lambda\tilde{H^1},
$$


(Actually, diagonalising the effective
Hamiltonian up to first order will include some corrections of second order,
since the perturbation is mixing the initial states, and giving non-diagonal elements
in $H_{eff}^1$. The energy correction will depend on $\lambda^2$ if $V=\lambda H^1$.)


\textbf{Second order}

After expanding the perturbation up to second order for the eigenstates of the system
we can arrive at the second order term of the effective Hamiltonian, where we have
separated the system into two substates $A$ and $B$ expanded by $\{\ket{\psi_i}\}_i$ and
$\{\ket{\psi_\mu}\}_\mu$, respectively.

$$
\tilde{H^2}_{i,j} = \sum_\mu\frac{
\bra{\psi_i} H^1 \ket{\psi_\mu}
\bra{\psi_\mu} H^1 \ket{\psi_j}
}
{\epsilon_j-\epsilon_\mu
}
$$

This correction is derived for the degenerate case of the initial states
in the small system expanded by the states $\ket{i}$.
Then, it must be symmetrized for the general
``not-only'' degenerate case, when $\epsilon_i\ne \epsilon_j$.

\begin{align}
\tilde{H^2}_{i,j} &= \sum_\mu\frac{1}{2}
\left(
\frac{
\bra{\psi_i} H^1 \ket{\psi_\mu}
\bra{\psi_\mu} H^1 \ket{\psi_j}}{\epsilon_j-\epsilon_\mu}
+
\frac{
\bra{\psi_i} H^1 \ket{\psi_\mu}
\bra{\psi_\mu} H^1 \ket{\psi_j}}{\epsilon_i-\epsilon_\mu}
\right)
\\
\tilde{H^2}_{i,j} &= \sum_\mu\frac{1}{2}
\bra{\psi_i} H^1 \ket{\psi_\mu}
\bra{\psi_\mu} H^1 \ket{\psi_j}
\left(\frac{1}{\epsilon_j-\epsilon_\mu}
+\frac{1}{\epsilon_i-\epsilon_\mu}
\right)
\end{align}


\textbf{Effective Hamiltonian}

The effective Hamiltonian up to second order is

\begin{align}
H_{eff}^2 &= \tilde{H^0} + \lambda \tilde{H^1} + \lambda^2\tilde{H^2}\\
(H_{eff}^2)_{i,j} &= \delta_{i,j}\epsilon_i +
\lambda \bra{\psi_i} H^1 \ket{\psi_j} \\
&+ 
\sum_\mu\frac{1}{2}
\bra{\psi_i} H^1 \ket{\psi_\mu}
\bra{\psi_\mu} H^1 \ket{\psi_j}
\left(\frac{1}{\epsilon_j-\epsilon_\mu}
+\frac{1}{\epsilon_i-\epsilon_\mu}
\right)
\end{align}

\textbf{kpm expansion}

The second order contribution to the effective Hamiltonian can be expressed as
\begin{align}
\tilde{H^2}_{i,j} &= \frac{1}{2}
\bra{\psi_i} H^1
\left[
\sum_\mu
\frac{\ket{\psi_\mu} \bra{\psi_\mu}}{\epsilon_j-\epsilon_\mu}
\right]
H^1 \ket{\psi_j}
+
\frac{1}{2}
\bra{\psi_i} H^1
\left[
\sum_\mu
\frac{\ket{\psi_\mu} \bra{\psi_\mu}}{\epsilon_i-\epsilon_\mu}
\right]
H^1 \ket{\psi_j}
\\
&= \frac{1}{2}
\bra{\psi_i} H^1
\left[P_B
\frac{1}{\epsilon_j-H^0}
P_B\right]
H^1 \ket{\psi_j}
+
\frac{1}{2}
\bra{\psi_i} H^1
\left[P_B
\frac{1}{\epsilon_i-H^0}
P_B\right]
H^1 \ket{\psi_j},
\end{align}

where $P_B$ is a projector over the space $B$ defined by $\{\psi_\mu\}_\mu$,
and we see the action of a Green's function defined as

$$
G(\epsilon, H^0) = \frac{1}{\epsilon-H^0}.
$$

This function can be expanded with the KPM, with the vectors of the expansion
being

$$
\ket{\phi_i} = P_B H^1\ket{\psi_i},
$$


such that


\begin{align}
\tilde{H^2}_{i,j} &= \frac{1}{2}
\bra{\phi_i} G(\epsilon_j, H^0) \ket{\phi_j}
+
\frac{1}{2}
\bra{\phi_i} G(\epsilon_i, H^0) \ket{\phi_j}
\end{align}

\textbf{Green's function expansion with KPM}

A function of a parameter and a Hamiltonian is expressed as 

$$
f(\epsilon, H) = \sum_k f(\epsilon, E_k) \ket{\psi_k}\bra{\psi_k},
$$

where the funcion $f$ is expanded in KPM as

\begin{align}
f(\epsilon, E_k) &= \frac{2}{\pi}\sum_m c_m(\epsilon)T_m(E_k)\\
c_m(\epsilon) &=\frac{1}{1+\delta_{m,0}}
\int_{-1}^1 \frac{f(\epsilon, E_k) T_m(E_k)}{\sqrt{1-E_k^2}}\mathrm{d}E_k.
\end{align}

To simplify the calculations we can instead expand

\begin{align}
\sqrt{1-E_k^2}f(\epsilon, E_k) &= \frac{2}{\pi}\sum_m c_m(\epsilon)T_m(E_k)\\
c_m(\epsilon) &=\frac{1}{1+\delta_{m,0}}
\int_{-1}^1 {f(\epsilon, E_k) T_m(E_k)}\mathrm{d}E_k.
\end{align}


The solution is expanded elsewhere\cite{Garcia2014} for the rescaled
Hamiltonian $\tilde{H}=(H-bI)/a$, and rescaled energies
$\tilde{\epsilon}=(\epsilon-b)/a$.

\begin{align}
f(\epsilon, H)^\pm &= \frac{1}{\epsilon-H \pm i0}\\
f(\epsilon, H)^\pm &= \frac{1}{\tilde{\epsilon}a+b-\tilde{H}a-b \pm i0}\\
f(\epsilon, H)^\pm &= a^{-1}\frac{1}{\tilde{\epsilon}-\tilde{H} \pm i0}\\
f(\epsilon, H)^\pm_M &= \mp \frac{2i}{a\sqrt{1-\epsilon^2}}
\sum_{m=0}^{M}\frac{g_m}{1+\delta_{m,0}}
\exp(\pm i\,m\,\mathrm{arccos}(\epsilon)) T_m(H)
\end{align}


% In our case, we can solve the last expression as
% 
% \begin{align}
% f(\epsilon, E) &= \frac{1}{\epsilon-E}\\
% T_m(E) &= \cos (m\,\mathrm{arcos}(E))\\
% \phi &= \mathrm{arcos}(E)\\
% \Rightarrow 
% {\mathrm{d}E}
% &= {\mathrm{d}\cos(\phi)} = -{\sin(\phi)} \mathrm{d}\phi\\
% &\rightarrow\\
% c_m(\epsilon) &=
% \frac{1}{1+\delta_{m,0}}
% \int_0^{\pi}
% {f(\epsilon, \cos(\phi)) \cos (m\,\phi)} {\sin(\phi)}
% \mathrm{d}\phi\\
% c_m(\epsilon) &=
% \frac{1}{1+\delta_{m,0}}
% \int_0^{\pi}
% \frac{\cos (m\,\phi)\sin(\phi)}{\epsilon-\cos(\phi)}
% \mathrm{d}\phi\\
% \theta &= \epsilon - \cos(\phi)\\
% \Rightarrow \mathrm{d}\theta &= -\mathrm{d}\cos(\phi)
% \\
% &= \sin(\phi)\mathrm{d}\phi


% \section{Introduction}

% \subsection{}

% \begin{figure}[!h]
% \centering
% \includegraphics[width=\linewidth]{name}
% \caption{}
% \label{fig1}
% \end{figure}

\begin{thebibliography}{2}
\bibitem{Garcia2014}Real-space calculation of the conductivity tensor for disordered topological matter
Jose H. Garcia, Lucian Covaci, Tatiana G. Rappoport. Phys. Rev. Lett. 114, 116602 (2015)
\end{thebibliography}


\end{document}