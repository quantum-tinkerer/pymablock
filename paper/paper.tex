% Created with jtex v.1.0.12
\documentclass[submission, Codebases]{SciPost}

% Change finalizecache to frozencache to compile using minted cache
% \usepackage[finalizecache,cachedir=minted-cache]{minted}
\usepackage{minted}
\usepackage{framed}
\usepackage{graphicx}
\usepackage[super]{nth}
\usepackage{orcidlink}
\usepackage{glossaries}
\usepackage{todonotes}
\usepackage{hyperref}
\usepackage{braket}
\makeglossaries

\binoppenalty=10000
\relpenalty=10000

\hypersetup{
    colorlinks,
    linkcolor={red!50!black},
    citecolor={blue!50!black},
    urlcolor={blue!80!black}
}

\definecolor{orcidlogocol}{HTML}{A6CE39}

\urlstyle{sf}

\DeclareSymbolFont{usualmathcal}{OMS}{cmsy}{m}{n}
\DeclareSymbolFontAlphabet{\mathcal}{usualmathcal}
\definecolor{bg}{rgb}{0.95,0.95,0.95}
\setminted{fontsize=\fontsize{9.5}{11}\selectfont, bgcolor=bg, baselinestretch=1, breaklines=true}
\usemintedstyle{perldoc}

\newcommand{\co}{\paragraph}
% Uncomment the following line to have comments appear in the PDF
% \newcounter{CommentNumber}
% \renewcommand{\co}[1]{\stepcounter{CommentNumber}\belowpdfbookmark{#1}{\arabic{CommentNumber}}}

\begin{document}

\begin{center}
{\Large \textbf{Pymablock}}
\end{center}

\begin{center}
Isidora~Araya~Day\textsuperscript{1, 2 $\star$}\orcidlink{0000-0002-2948-4198},
Sebastian~Miles\textsuperscript{1},
Hugo~K.~Kerstens\textsuperscript{2}\orcidlink{0009-0003-9685-5088},
Daniel~Varjas\textsuperscript{3}\orcidlink{0000-0002-3283-6182},
Anton~R.~Akhmerov\textsuperscript{2 $\dagger$}\orcidlink{0000-0001-8031-1340}
\end{center}

\begin{center}
{\bf 1} QuTech, Delft University of Technology, Delft 2600 GA, The Netherlands \\
{\bf 2} Kavli Institute of Nanoscience, Delft University of Technology, 2600 GA Delft, The Netherlands \\
{\bf 3} Max Planck Institute for the Physics of Complex Systems, Nöthnitzer Strasse 38, 01187 Dresden, Germany \\
${}^\star$ {\small \sf iarayaday@gmail.com}
${}^\dagger${\small \sf pymablock@antonakhmerov.org}
\end{center}

\begin{center}
    \today
\end{center}

\section*{Abstract}
\textbf{
A common problem in studies of complex quantum-mechanical systems is to reduce the number of degrees of freedom and construct an effective Hamiltonian of the smaller subspace by using quasi-degenerate perturbation theory.
While the Schrieffer--Wolff transformation solves this problem, its scaling is suboptimal and implementing it efficiently is both challenging and error-prone.
We introduce an algorithm for solving this problem as well as a Python package, Pymablock, that implements it.
Our algorithm combines an optimal asymptotic scaling with a range of other improvements.
The package supports numerical and analytical calculations of any order and is designed to be interoperable with any other packages for specifying the Hamiltonian.
We demonstrate how the package handles constructing a k.p model, analyses a superconducting qubit, and computes the low-energy spectrum of a large tight-binding model and compare its performance with reference calculations.
}

\vspace{10pt}
\noindent\rule{\textwidth}{1pt}
\tableofcontents\thispagestyle{fancy}
\noindent\rule{\textwidth}{1pt}
\vspace{10pt}

\listoftodos
\section{Introduction}

\co{Effective models enable the study of complex physical systems by reducing the space of interest to a low-energy one.}
Effective models enable the study of complex quantum systems by reducing the dimensionality of the Hilbert space.
Their construction separates the low and high-energy subspaces by block-diagonalizing a perturbed Hamiltonian
%
\begin{equation}
    \mathcal{H} = \begin{pmatrix}H_0^{AA} & 0 \\ 0 & H_0^{BB}\end{pmatrix} + \mathcal{H}',
\end{equation}
%
where $H_0^{AA}$ and $H_0^{BB}$ are separated by an energy gap, and $\mathcal{H}'$ is a series in a perturbative parameter.
This procedure requires finding a series of the basis transformation $\mathcal{U}$ that is unitary and that also cancels the off-diagonal block of the transformed Hamiltonian order by order, as shown in Fig.~\ref{fig:block_diagonalization}.
The low-energy effective Hamiltonian $\tilde{\mathcal{H}}^{AA}$ is then a series in the perturbative parameter, whose eigenvalues and eigenvectors are approximate solutions of the complete Hamiltonian.
As a consequence, the effective model is sufficient to describe the low-energy properties of the original system while also being simpler and easier to handle.

\co{A standard approach to constructing the effective model is the Schrieffer-Wolff algorithm.}
A common approach to constructing an effective Hamiltonian is the Schrieffer--Wolff transformation~\cite{Schrieffer_1966,Bravyi_2011}, also known as Löwdin partitioning~\cite{Lowdin_1962}, or quasi-degenerate perturbation theory.
This method parameterizes the unitary transformation $\mathcal{U} = e^{-\mathcal{S}}$ and finds the series $\mathcal{S}$ that decouples the $A$ and $B$ subspaces of $\tilde{\mathcal{H}} = e^{\mathcal{S}}\mathcal{H}e^{-\mathcal{S}}$.
This idea enabled advances in multiple fields of quantum physics.
As an example, all the k.p models are a result of treating crystalline momentum as a perturbation that only weakly mixes atomic orbitals separated in energy~\cite{Luttinger_1955}.
More broadly, this method serves as a go-to tool in the study of superconducting circuits and quantum dots, where couplings between circuit elements and drives are treated as perturbations to reproduce the dynamics of the system~\cite{Krantz_2019,Romhanyi_2015}.
%
\begin{figure}[h!]
    \centering
    \includegraphics[width=\textwidth]{figures/diagrams_H.pdf}
    \caption{
      Block-diagonalization of a Hamiltonian with a first order perturbation.
    }
    \label{fig:block_diagonalization}
\end{figure}

\co{Even though these methods are standard, their algorithm is computationally expensive, scaling poorly for large systems and high orders.}
Constructing effective Hamiltonians is, however, both algorithmically complex and computationally expensive.
This is a consequence of the recursive equations that define the unitary transformation, which require an exponentially growing number of matrix products in each order.
In particular, already a 4-th order perturbative expansion that is necessary for many applications may require hundreds of terms.
While the computational complexity is only a nuisance when analysing model systems, it becomes a bottleneck whenever the Hilbert space is high-dimensional.
Several alternative approaches improve the performance of the Schrieffer--Wolff algorithm by either using different parametrizations of the unitary transformation~\cite{Van_Vleck_1929, Shavitt_1980, Lowdin_1962, Klein_1974, Suzuki_1983}, adjusting the problem setting to density matrix perturbation theory~\cite{McWeeny_1962, Truflandier_2020}, or a finding a similarity transform instead of a unitary~\cite{Bloch_1958}.
A more recent line of research even applies the ideas of Schrieffer--Wolff transformation to quantum algorithms for the study of many-body systems~\cite{Wurtz_2020, Zhang_2022}.
Despite these advances, neither of the approaches combines an optimal scaling with the ability to construct effective Hamiltonians.

\co{We develop an efficient algorithm capable of symbolic and numeric computations and make it available in Pymablock.}
We introduce an algorithm to construct effective models with optimal scaling, thus making it possible to find high order effective models for systems with millions of degrees of freedom.
This algorithm exploits the efficiency of recursive evaluations of series satisfying polynomial constraints and obtains the same effective Hamiltonian as the Schrieffer--Wolff transformation.
We make the algorithm available via the open source package Pymablock \footnote{The documentation and tutorials are available in \url{https://pymablock.readthedocs.io/}}(PYthon MAtrix BLOCK-diagonalization), a versatile tool for the study of numerical and symbolic models.

\section{Constructing an effective model}
\subsection{Pymablock workflow}

\co{The workflow of Pymablock consists of three steps.}
Building an effective model using Pymablock is a three step process:
%
\begin{itemize}
\item Define a Hamiltonian
\item Call \mintinline{python}|pymablock.block_diagonalize|
\item Request the desired order of the effective Hamiltonian
\end{itemize}
%
The following code snippet shows how to use Pymablock to compute the fourth
order correction to an effective Hamiltonian $\tilde{\mathcal{H}}$:
%
\begin{minted}{ipython}
from pymablock import block_diagonalize

# Define perturbation theory
H_tilde, *_ = block_diagonalize([H_0, H_1], subspace_eigenvectors=[vecs_A, vecs_B])

# Request 4th order correction to the effective Hamiltonian
H_AA_4 = H_tilde[0, 0, 4]
\end{minted}

\co{Depending on the input Hamiltonian, Pymablock uses specific routines to find
the effective model, so that symbolic expressions are compact and numerics are
efficient.}

The function \mintinline{python}|block_diagonalize| interprets the Hamiltonian $H_0 +
H_1$ and calls the block diagonalization routines depending on the
type and sparsity of the input.
This is the main function of Pymablock, and it is the only one that the user
ever needs to call.
It first output is a multivariate series whose terms are different blocks and
orders of the effective Hamiltonian.
Calling \mintinline{python}|block_diagonalize| is not computationally expensive, because the
terms of the series are only computed when requested.

\subsection{k.p model of bilayer graphene}

\co{We use bilayer graphene to illustrate how to use Pymablock with analytic models.}

To illustrate how to use Pymablock with analytic models, we consider two layers
of graphene stacked on top of each other.
Our goal is to find the low energy model near the $\mathbf{K}$ point
\cite{McCann_2013}.
First, we construct the Hamiltonian of bilayer graphene from its tight-binding
model.

\begin{figure}[!htbp]
\centering
\includegraphics[width=0.3125\linewidth]{figures/bilayer_graphene.pdf}
\caption[]{Crystal structure and hoppings of bilayer graphene.}
\label{bilayer}
\end{figure}

The main features of the model are:
%
\begin{itemize}
\item The unit cell is spanned by vectors $\mathbf{a}_1 = (1/2, \sqrt{3}/2)$ and $\mathbf{a}_2=( -1/2, \sqrt{3}/2)$.
\item The unit cell contains 4 atoms with wave functions $\phi_{A,1}, \phi_{B,1}, \phi_{A,2}, \phi_{B,2}$.
\item The hoppings within each layer are $t_1$.
\item The hopping between atoms that are on top of each other is $t_2$.
\item The layers have an onsite potential $\pm m$.
\end{itemize}

\subsubsection{Defining a symbolic Hamiltonian}

We define the Bloch Hamiltonian using the Sympy package for symbolic Python
\cite{Meurer_2017}.
%
\begin{minted}{ipython}
import numpy as np
from sympy import symbols, Matrix, sqrt, Eq, exp, I, pi, Add, MatAdd
from sympy.physics.vector import ReferenceFrame

t_1, t_2, m = symbols("t_1 t_2 m", real=True)
alpha = symbols(r"\alpha")

H = Matrix([
    [m, t_1 * alpha, 0, 0],
    [t_1 * alpha.conjugate(), m, t_2, 0],
    [0, t_2, -m, t_1 * alpha],
    [0, 0, t_1 * alpha.conjugate(), -m]]
)
Eq(symbols("H"), H, evaluate=False)
\end{minted}

\begin{minted}{ipython}
Eq(H, Matrix([
[                    m, \alpha*t_1,                     0,          0],
[t_1*conjugate(\alpha),          m,                   t_2,          0],
[                    0,        t_2,                    -m, \alpha*t_1],
[                    0,          0, t_1*conjugate(\alpha),         -m]]))
\end{minted}
%
where $\alpha(\mathbf{k}) = 1 + e^{i \mathbf{k'} \cdot (\mathbf{a}_1 +
\mathbf{a}_2)}$ and $\mathbf{k'} = (4\pi/3 + k_x, k_y)$, because we choose
$\mathbf{K}=(4\pi/3, 0)$ as the reference point for the k.p effective model.

\subsubsection{Defining the perturbative series}

%  **We define the perturbative series**

To call \mintinline{python}|block_diagonalize|, we use the eigenvectors of the unperturbed
Hamiltonian at the $\mathbf{K}$ point.
To demonstrate the capabilities of Pymablock, we use $m$ as a perturbative
parameter too.
The unperturbed Hamiltonian is then $H(\alpha(\mathbf{K}) = m = 0)$, and its
eigenvectors are:
%
\begin{align}
v_{A,1} &= \begin{pmatrix} 1 \\ 0 \\ 0 \\ 0 \end{pmatrix} &
v_{A,2} &= \begin{pmatrix} 0 \\ 1 \\ 0 \\ 1 \end{pmatrix} &
v_{B,1} &= \frac{1}{\sqrt{2}} \begin{pmatrix} 0 \\ 0 \\ -1 \\ 1 \end{pmatrix} &
v_{B,2} &= \frac{1}{\sqrt{2}} \begin{pmatrix} 0 \\ 0 \\ 1 \\ 1 \end{pmatrix}
\end{align}
%
These determine the basis on which the perturbative corrections are computed
and $A$, the subspace of interest for the effective model.
Then, we substitute $\alpha(\mathbf{k})$ into the Hamiltonian, and define the
block diagonalization routine using that $k_x$, $k_y$, and $m$ are perturbative
parameters.
%
\begin{minted}{ipython}
from pymablock import block_diagonalize

H_tilde, U, U_adjoint = block_diagonalize(
    H.subs({alpha: alpha_k}),
    symbols=(k_x, k_y, m),
    subspace_eigenvectors=[vecs[:, :2], vecs[:, 2:]]
)
\end{minted}
%
The order of the variables in the perturbative series will be that of \mintinline{python}{symbols}.

\subsubsection{Requesting the effective Hamiltonian}

We need corrections up to third order in momentum to compute the standard
quadratic dispersion of bilayer graphene and trigonal warping.
Therefore, we define second and third order terms in momentum and group them
total power of momentum.
%
\begin{minted}{ipython}
k_square = np.array([[0, 1, 2], [2, 1, 0]])
k_cube = np.array([[0, 1, 2, 3], [3, 2, 1, 0]])
\end{minted}
%
Querying \mintinline{python}{H\_tilde} returns the results in a masked numpy array, so we
define \mintinline{python}{H\_tilde\_AA} to gather different entries into one symbolic expression.
Finally, the result is a symbolic expression of the effective Hamiltonian.
%
\begin{minted}{ipython}
mass_term = H_tilde_AA([0], [0], [1])
kinetic = H_tilde_AA(*k_square, 0)
mass_correction = H_tilde_AA(*k_square, 1)
cubic = H_tilde_AA(*k_cube, 0)
MatAdd(mass_term + kinetic, mass_correction + cubic, evaluate=False)
\end{minted}
%
\begin{minted}{ipython}
Matrix([
[                                                m, 3*t_1**2*( -k_x**2 - 2*I*k_x*k_y + k_y**2)/(4*t_2)],
[3*t_1**2*( -k_x**2 + 2*I*k_x*k_y + k_y**2)/(4*t_2),                                                -m]]) + Matrix([
[                                    3*m*t_1**2*( -k_x**2 - k_y**2)/(2*t_2**2), sqrt(3)*t_1**2*(k_x**3 - 5*I*k_x**2*k_y + 9*k_x*k_y**2 + 3*I*k_y**3)/(8*t_2)],
[sqrt(3)*t_1**2*(k_x**3 + 5*I*k_x**2*k_y + 9*k_x*k_y**2 - 3*I*k_y**3)/(8*t_2),                                      3*m*t_1**2*(k_x**2 + k_y**2)/(2*t_2**2)]])
\end{minted}
%
The first term contains the standard quadratic dispersion of bilayer graphene
with a gap.
The second term contains trigonal warping and the coupling between the gap and
momentum.

\subsection{Induced gap in a double quantum dot}

\co{Large systems pose an additional challenge due to the scaling of linear
algebra routines for large matrices.}
Large systems pose an additional challenge due to the cubic scaling of linear algebra
routines on matrices' size.
Pymablock handles large systems by using sparse matrices and avoiding the
construction of the full Hamiltonian.
We illustrate its efficiency with a model of two quantum dots coupled to a
superconductor between them.

\textit{(Include figure with scheme of the system)}

\subsubsection{Building the Hamiltonian with Kwant}

\co{We use Kwant to build the Hamiltonian of the system.}
We use the Kwant package \cite{Groth_2014} to build
the Hamiltonian of the system.
In the following code, we define a square lattice of $L \times W = 200 \times
40$ sites with 2 orbitals per unit cell.
The lattice is divided into three regions: a quantum dot on the left, a
superconducting region in the middle, and a quantum dot on the right.
%
\begin{minted}{ipython}
import tinyarray as ta
import matplotlib.backends
import scipy.linalg
from scipy.sparse.linalg import eigsh
import numpy as np
import kwant
import matplotlib.pyplot as plt
color_cycle = ["#5790fc", "#f89c20", "#e42536"]

from pymablock import block_diagonalize


sigma_z = ta.array([[1, 0], [0, -1]], float)
sigma_x = ta.array([[0, 1], [1, 0]], float)

syst = kwant.Builder()
lat = kwant.lattice.square(norbs=2)
L, W = 200, 40

def normal_onsite(site, mu_n, t):
    return ( -mu_n + 4 * t) * sigma_z

def sc_onsite(site, mu_sc, Delta, t):
    return ( -mu_sc + 4 * t) * sigma_z + Delta * sigma_x

syst[lat.shape((lambda pos: abs(pos[1]) < W and abs(pos[0]) < L), (0, 0))] = normal_onsite
syst[lat.shape((lambda pos: abs(pos[1]) < W and abs(pos[0]) < L / 3), (0, 0))] = sc_onsite
syst[lat.neighbors()] = lambda site1, site2, t: -t * sigma_z

def barrier(site1, site2):
    return (abs(site1.pos[0]) - L / 3) * (abs(site2.pos[0]) - L / 3) < 0

syst[(hop for hop in syst.hoppings() if barrier(*hop))] = (
    lambda site1, site2, t_barrier: -t_barrier * sigma_z
)
\end{minted}

The chemical potentials of the normal and superconducting regions are $\mu_n$
and $\mu_{sc}$, respectively, $\Delta$ is the superconducting gap, and $t$
is the hopping amplitude within each region.
The barrier strength between the quantum dots and the superconductor is
$t_{barrier}$, a parameter that we treat as a perturbation.
We will also consider the asymmetry of the dot potentials, $\delta \mu$, as a
perturbation.
%
\textbf{(Include figure with the system)}
%
The system is large: with this many sites even storing all the eigenvectors
would take 60 GB of memory.
Therefore, we use sparse matrices and compute only a few eigenvectors.
In this case, perturbation theory allows us to compute the effective
Hamiltonian of the low energy degrees of freedom.
%
To get the unperturbed Hamiltonian, we use the following values for $\mu_n$,
$\mu_{sc}$, $\Delta$, $t$, and $t_{\text{barrier}}$.
%
\begin{minted}{ipython}
params = dict(
    mu_n=0.05,
    mu_sc=0.3,
    Delta=0.05,
    t=1.,
    t_barrier=0.,
)

h_0 = syst.hamiltonian_submatrix(params=params, sparse=True).real
\end{minted}

The barrier strength and the asymmetry of the dot potentials are the two
perturbations that we vary.

\begin{minted}{ipython}
barrier = syst.hamiltonian_submatrix(
    params={**{p: 0 for p in params.keys()}, "t_barrier": 1}, sparse=True
).real
delta_mu = (
    kwant.operator.Density(syst, (lambda site: sigma_z * site.pos[0] / L)).tocoo().real
)
\end{minted}

\subsubsection{Define the perturbative series}

Since the Hamiltonian is large and we are only interested in the low energy
subspace, it is sufficient to compute the 4 lowest eigenvectors of the
unperturbed Hamiltonian.
These are the lowest energy Andreev states in two quantum dots.
%
\begin{minted}{ipython}
# %%time

vals, vecs = eigsh(h_0, k=4, sigma=0)
vecs, _ = scipy.linalg.qr(vecs, mode="economic")  # orthogonalize the vectors
\end{minted}
%
To orthogonalize the eigenvectors manually because
\mintinline{python}{scipy.sparse.linalg.eigsh} does not return orthogonal eigenvectors if the
matrix is complex and eigenvalues are degenerate.
%
We now define the block diagonalization routine and compute the few lowest
orders of the effective Hamiltonian.
Here we only provide the set of vectors of the interesting subspace.
This selects the \mintinline{python}{pymablock.implicit} method that uses efficient sparse
solvers for Sylvester's equation.
%
\begin{minted}{ipython}
# %%time

H_tilde, *_ = block_diagonalize([h_0, barrier, delta_mu], subspace_eigenvectors=[vecs])
\end{minted}
%
Block diagonalization is now the most time consuming step because it requires
pre-computing several decompositions of the full Hamiltonian.
It is, however, manageable and it only produces a constant overhead.

\subsubsection{Get the effective Hamiltonian}

For convenience, we collect the lowest three orders in each parameter in an
appropriately sized tensor.
%
\begin{minted}{ipython}
# %%time

# Combine all the perturbative terms into a single 4D array
fill_value = np.zeros((), dtype=object)
fill_value[()] = np.zeros_like(H_tilde[0, 0, 0, 0])
h_tilde = np.array(np.ma.filled(H_tilde[0, 0, :3, :3], fill_value).tolist())
\end{minted}
%
We see that we have obtained the effective model in only a few seconds.
We can now compute the low energy spectrum after rescaling the perturbative
corrections by the magnitude of each perturbation.
%
\begin{minted}{ipython}
def effective_energies(h_tilde, barrier, delta_mu):
    barrier_powers = barrier ** np.arange(3).reshape( -1, 1, 1, 1)
    delta_mu_powers = delta_mu ** np.arange(3).reshape(1, -1, 1, 1)
    return scipy.linalg.eigvalsh(
        np.sum(h_tilde * barrier_powers * delta_mu_powers, axis=(0, 1))
    )
\end{minted}
%
Finally, we plot the spectrum
%
\begin{minted}{ipython}
barrier_vals = np.array([0, 0.5, .75])
delta_mu_vals = np.linspace(0, 10e -4, num=101)
results = [
    np.array([effective_energies(h_tilde, bar, dmu) for dmu in delta_mu_vals])
    for bar in barrier_vals
]

plt.figure(figsize=(10, 6), dpi=200)
[
    plt.plot(delta_mu_vals, result, color=color, label=[f"$t_b={barrier}$"] + 3 * [None])
    for result, color, barrier in zip(results, color_cycle, barrier_vals)
]
plt.xlabel(r"$\delta_\mu$")
plt.ylabel(r"$E$")
plt.legend();
\end{minted}
%
% \includegraphics[width=0.7\linewidth]{files/95fc712be507bbaddfe033b24c38d25d.png}
%
As expected, the crossing at $E=0$ due to the dot asymmetry is lifted when the
dots are coupled to the superconductor. In addition, we observe how the
proximity gap of the dots increases with the coupling strength.
%
We also see that computing the spectrum perturbatively is faster than
repeatedly using sparse diagonalization for a set of parameters.
In this example the total runtime of Pymablock would only allow us to compute
the  eigenvectors at around 5 points in the parameter space.

\section{Algorithm for block diagonalization}

\subsection{Problem statement}

Pymablock finds a series of the unitary transformation $\mathcal{U}$ (we use
calligraphic letters to denote series) that block-diagonalizes the Hamiltonian
%
\begin{align}
\label{hamiltonian}
\mathcal{H} = H_0 + \mathcal{H}',\quad H_0 = \begin{pmatrix}
H_0^{AA} & 0\\
0 & H_0^{BB}
\end{pmatrix},
\end{align}
%
with $\mathcal{H}'$ containing an arbitrary number and orders of perturbations.
The series here may be multivariate, and they represent sums of the form
%
\begin{align}
\mathcal{A} = \sum_{n_1=0}^\infty \sum_{n_2=0}^\infty \cdots \sum_{n_k=0}^\infty \lambda_1^{n_1} \lambda_2^{n_2} \cdots \lambda_k^{n_k} A_{n_1, n_2, \ldots, n_k},
\end{align}
%
where $\lambda_i$ are the perturbation parameters and $A_{n_1, n_2, \ldots,
n_k}$ are linear operators.
%
The problem statement, therefore, is finding $\mathcal{U}$ and
$\tilde{\mathcal{H}}$ such that:
%
\begin{align}
\label{eq:problem_definition}
\tilde{\mathcal{H}} = \mathcal{U}^\dagger \mathcal{H} \mathcal{U},\quad \tilde{\mathcal{H}}^{AB} = 0,\quad \mathcal{U}^\dagger \mathcal{U} = 1,
\end{align}
%
where series multiply according to the Cauchy product:
%
$$
\mathcal{C} = \mathcal{A}\mathcal{B} \Leftrightarrow C_\mathbf{n} = \sum_{\mathbf{m} + \mathbf{p} = \mathbf{n}} A_\mathbf{m} B_\mathbf{p}.
$$
%
This product is the most expensive operation in perturbation theory, because it
involves a large number of multiplications between potentially large matrices.
For example, evaluating $\mathbf{n}$-th order of $\mathcal{C}$ requires
$\sim\prod_i n_i = N$ multiplications of the series elements.
A direct computation of all the possible index combinations in a product
between three series $\mathcal{A}\mathcal{B}\mathcal{C}$ would have a higher
cost $\sim N^2$, however if we use associativity of the product and compute
this as $(\mathcal{A}\mathcal{B})\mathcal{C}$, then the scaling of the cost
stays $\sim N$.

There are many ways to solve this problem that give identical expressions for
$\mathcal{U}$ and $\tilde{\mathcal{H}}$.
We are searching for a procedure that satisfies two additional constraints:
%
\begin{itemize}
    \item It has the same complexity scaling as a Cauchy product.
    \item It does not require multiplications by $H_0$.
    This is because in perturbation theory, $n$-th order  corrections to
    $\tilde{\mathcal{H}}$ carry $n$ energy denominators $1/(E_i - E_j)$.
    Therefore, any additional multiplications by $H_0$ must cancel with
    additional energy denominators.
    Multiplying by $H_0$ is therefore unnecessary work, and it gives longer
    intermediate expressions.
\end{itemize}
%
The goal of our algorithm is thus to be efficient and to produce compact
results that do not require further simplifications.
\subsection{Existing solutions}
\co{Pymablock's algorithm does not use the Schrieffer-Wolff transformation,
because the former is inefficient.}
A common approach to construct effective Hamiltonians is to use a
Schrieffer-Wolff transformation:
%
\begin{align}
\tilde{\mathcal{H}} = e^\mathcal{S} &\mathcal{H} e^{-\mathcal{S}}, \\
e^{\mathcal{S}} = 1 + \mathcal{S} + \frac{1}{2!} \mathcal{S} \mathcal{S}
+ &\frac{1}{3!} \mathcal{S} \mathcal{S} \mathcal{S} + \cdots,
\end{align}
%
where $\mathcal{S} = \sum_n S_n$ is an antihermitian polynomial series in the
perturbative parameter, making $e^\mathcal{S}$ a unitary transformation.
In this approach, $\mathcal{S}$ is found by ensuring unitarity and the block
diagonalization of the Hamiltonian to every order, a procedure that amounts to
solving a recursive equation whose terms are nested commutators between series.
Moreover, the transformed Hamiltonian is also given by a series of nested
commutators
%
\begin{equation}
\tilde{\mathcal{H}} = \sum_{j=0}^\infty \frac{1}{j!} \Big [\mathcal{H}, \sum_{n=0}^{\infty} S_n \Big ]^{(j)},
\end{equation}
%
a computationally expensive expression because it requires computing
exponentially many matrix products.
This expression also requires truncating the series at the same order
to which $\mathcal{S}$ is computed, which is a waste of computational resources.
Finally, generalizing the Schrieffer-Wolff transformation to multiple
perturbations is only straightforward if the perturbations are bundled
together.
However, this makes it impossible to request individual order combinations
of the perturbations, making it necessary to compute more terms than needed.

\co{There are algorithms that use different parametrizations for U a
difference that is crucial for efficiency, even though the results are
equivalent.}
The algorithm used to block diagonalize a Hamiltonian perturbatively is,
however, not unique.
Alternative parametrizations of the unitary transformation $\mathcal{U}$
require solving unitarity and block diagonalization conditions too, but
give rise to a different recursive procedure for the series elements.
For example, using hyperbolic functions
%
\begin{gather}
\mathcal{U} = \cosh{\mathcal{G}} + \sinh{\mathcal{G}}, \quad
\mathcal{G} = \sum_{i=0}^{\infty} G_i,
\end{gather}
%
leads to recursive equations for $G_i$ that use Bernoulli numbers as
prefactors for contributions from lower order terms \cite{Shavitt_1980},
making the algorithm inefficient.
On the other hand, using a polynomial series directly
%
\begin{align}
\mathcal{U} &= \sum_{i=0}^{\infty} U_i,
\end{align}
%
gives rise to a recursive equation for $U_i$ that is free from any additional
coefficients
\cite{Van_Vleck_1929}, \cite{Lowdin_1962}
\cite{Klein_1974}, \cite{Suzuki_1983}.
This choice, however, results in an expression for $\tilde{\mathcal{H}}$ whose
terms include products by $H_0$, and therefore requires additional
simplifications.

\co{The existing algorithms with linear scaling are not suitable for the
construction of an effective Hamiltonian.}
Despite the conceptual equivalence of the algorithms and the agreement of their
results, there is a crucial difference in their computational efficiency: a
Schrieffer-Wolff transformation has an exponential scaling with the
perturbative order, but it can be reduced.
C. Bloch's perturbation theory \cite{Bloch_1958}, for example,
aims to find a similarity transform that brings the Hamiltonian to a
block-triangular form, losing its Hermiticity and orhogonality properties.
The iterative procedure to find the similarity transform requires computing
fewer expressions for the series elements than the Schrieffer-Wolff
transformation \cite{Bravyi_2011}, and the effective Hamiltonian is more
compact.
However, the algorithm is only useful to obtain the spectrum of the low energy
subspace, not its wavefunctions.
Reference \cite{Li_2022}, for example, introduces an
algorithm with linear scaling for diagonalization of a single state by
reformulating the recursive steps of the Schrieffer-Wolff transformation.
Block diagonalization of a Hamiltonian, however, recovers the exponential
scaling.
Another approach with linear scaling is that of density matrix perturbation
theory \cite{McWeeny_1962,McWeeny_1968,Truflandier_2020}, in which the
density matrix of a single-particle system is also a power series with
respect to a perturbative parameter:
%
\begin{align}
  \mathcal{D} = \sum_{i=0}^{\infty} D_i.
\end{align}
%
The elements of the series are found by solving two recursive conditions,
$\mathcal{D}^2 = \mathcal{D}$ and $[\mathcal{H}, \mathcal{D}]=0$.
This approach, however, does not provide an effective Hamiltonian, and even
though it has a linear scaling, it deals with the entire Hilbert space, not
only the low energy subspace.
We thus identify the need for an algorithm that combines linear scaling with
the ability to construct an effective Hamiltonian.

\subsection{Our solution}

To find $\mathcal{U}$, let us separate it into an identity and $\mathcal{U}' =
\mathcal{W} + \mathcal{V}$:
%
\begin{align}
\label{eq:U}
\mathcal{U} = 1 + \mathcal{U}' = 1 + \mathcal{W} + \mathcal{V},\quad \mathcal{W}^\dagger = \mathcal{W},\quad \mathcal{V}^\dagger = -\mathcal{V}.
\end{align}
%
First, we use the unitarity condition
$\mathcal{U}^\dagger \mathcal{U} = 1$ by substituting $\mathcal{U}'$ into it
and obtain
%
\begin{align}
\label{eq:W}
\mathcal{W} &= \frac{1}{2}(\mathcal{U}'^\dagger + \mathcal{U}') \nonumber \\
  &= \frac{1}{2} \Big[(1 + \mathcal{U}'^\dagger)(1+\mathcal{U}') - 1 - \mathcal{U}'^\dagger \mathcal{U}' \Big] \nonumber \\
  &= -\frac{1}{2} \mathcal{U}'^\dagger \mathcal{U}'.
\end{align}
%
Because $\mathcal{U}'$ has no $0$-th order term, $(\mathcal{U}'^\dagger
\mathcal{U}')_\mathbf{n}$ does not depend on the $\mathbf{n}$-th order of
$\mathcal{U}'$ nor $\mathcal{W}$.
More generally, a Cauchy product $\mathcal{A}\mathcal{B}$ where $\mathcal{A}$
and $\mathcal{B}$ have no $0$-th order terms depends on $\mathcal{A}_1, \ldots,
\mathcal{A}_{n-1}$ and $\mathcal{B}_1, \ldots, \mathcal{B}_{n-1}$.
This allows us to use Cauchy products to define recurrence relations, which
we apply throughout the algorithm.
Therefore, we compute $\mathcal{W}$ as a Cauchy product of $\mathcal{U}'$ with
itself \footnotemark[1].

\footnotetext[1]{
Using the definition of $\mathcal{W}$ as the Hermitian part of $\mathcal{U}'$,
and the unitarity condition:
%
$$
2\mathcal{W}
= \mathcal{U}' + \mathcal{U}'^\dagger
= -\mathcal{U}'^\dagger \mathcal{U}'
= -\mathcal{W}^2 + \mathcal{V}^2.
$$
%
we see that we could alternatively define $\mathcal{W}$ as a Taylor series in
$\mathcal{V}$:
%
$$
\mathcal{W} = \sqrt{1 + \mathcal{V}^2} - 1 \equiv f(\mathcal{V}) \equiv \sum_n a_n \mathcal{V}^{2n},
$$
%
however the scaling of such a Cauchy product becomes slower if we need to
compute a Taylor expansion of a series:
%
$$
f(\mathcal{A}) = \sum_{n=0}^\infty a_n \mathcal{A}^n.
$$
%
However, evaluating a Taylor expansion of a given series has a higher scaling
of complexity.
A direct computation of all possible products of terms would require $\sim \exp
N$ multiplications.
We improve on this by defining a new series as $\mathcal{A}^{n+1} =
\mathcal{A}\mathcal{A}^{n}$ and reusing the previously computed results, which
brings these costs down to $\sim N^2$.
Using the Taylor expansion approach is therefore both more complicated and more
computationally expensive than the recurrent definition in \ref{eq:W}.
}

To compute $\mathcal{U}'$ we also need to find $\mathcal{V}$, which is defined
by the requirement $\tilde{\mathcal{H}}^{AB} = 0$.
Additionally, we constrain $\mathcal{V}$ to be block off-diagonal:
$\mathcal{V}^{AA} = \mathcal{V}^{BB} = 0$,
so that the resulting unitary transformation is equivalent to the
Schrieffer-Wolff transformation \footnotemark[2].
In turn, this means that $\mathcal{W}$ is block-diagonal and that the norm
of $\mathcal{U}'$ is minimal.

\footnotetext[2]{
\textbf{Equivalence to Schrieffer-Wolff transformation}
Both the Pymablock algorithm and the more commonly used Schrieffer-Wolff
transformation find a unitary transformation $\mathcal{U}$ such that
$\tilde{\mathcal{H}}^{AB}=0$.
They are therefore equivalent up to a gauge choice in each subspace, $A$ and
$B$.
We establish the equivalence between the two by demonstrating that this gauge
choice is the same for both algorithms.
The Schrieffer-Wolff transformation uses $\mathcal{U} = \exp \mathcal{S}$,
where $\mathcal{S} = -\mathcal{S}^\dagger$ and $\mathcal{S}^{AA} =
\mathcal{S}^{BB} = 0$.
The series $\exp\mathcal{S}$ contains all possible products of $S_n$ of all
lengths with fractional prefactors.
For every term $S_{k_1}S_{k_2}\cdots S_{k_n}$, there is a corresponding term
$S_{k_n}S_{k_{n-1}}\cdots S_{k_1}$ with the same prefactor.
If the number of $S_{k_n}$ is even, then both terms are block-diagonal since
each $S_n$ is block off-diagonal.
Because $S_n$ are anti-Hermitian, the two terms are Hermitian conjugates of each
other, and therefore their sum is Hermitian.
On the other hand, if the number of $S_{k_n}$ is odd, then the two terms are
block off-diagonal and their sum is anti-Hermitian by the same reasoning.
Therefore, just like in our algorithm, the diagonal blocks of $\exp S$ are
Hermitian, while off-diagonal blocks are anti-Hermitian.
Schrieffer-Wolff transformation produces a unique answer and satisfies the same
diagonalization requirements as our algorithm, which means that the two are
equivalent.
}

\co{We find V and the transformed Hamiltonian.}
To find $\mathcal{V}$, we need to first look at the transformed Hamiltonian:
%
\begin{align}
\tilde{\mathcal{H}} = \mathcal{U}^\dagger \mathcal{H} \mathcal{U} = H_0 +
\mathcal{U}'^\dagger H_0 + H_0 \mathcal{U}' + \mathcal{U}'^\dagger H_0
\mathcal{U}' + \mathcal{U}^\dagger\mathcal{H'}\mathcal{U},
\end{align}
%
where we used $\mathcal{U}=1+\mathcal{U}'$ and $\mathcal{H} = H_0 +
\mathcal{H'}$.
Because we want to avoid unnecessary products by $H_0$, we need to get rid of
the terms that contain it by replacing them with an alternative expression.
Our strategy is to define an auxiliary operator $\mathcal{X}$ that we can
compute without ever multiplying by $H_0$.
Like $\mathcal{U}'$, $\mathcal{X}$ needs to be defined via a recurrence
relation, which we will find later.
Because the expression above has $H_0$ multiplied by $\mathcal{U}'$ by the left
and by the right, we get rid of these terms by making sure that $H_0$
multiplies terms from one side only.
To achieve this, we choose $\mathcal{X}=\mathcal{Y}+\mathcal{Z}$ to be the commutator between
$\mathcal{U}'$ and $H_0$:
%
\begin{align}
\label{eq:XYZ}
\mathcal{X} \equiv [\mathcal{U}', H_0] = \mathcal{Y} + \mathcal{Z}, \quad
\mathcal{Y} \equiv [\mathcal{V}, H_0] = \mathcal{Y}^\dagger,\quad
\mathcal{Z} \equiv [\mathcal{W}, H_0] = -\mathcal{Z}^\dagger,
\end{align}
%
where $\mathcal{Y}$ is therefore block off-diagonal and $\mathcal{Z}$, block
diagonal.
We use $H_0 \mathcal{U}' = \mathcal{U}' H_0 -\mathcal{X}$ to move $H_0$ through
to the right and find
%
\begin{align}
\label{eq:H_tilde}
  \tilde{\mathcal{H}}
  &= H_0 + \mathcal{U}'^\dagger H_0 + (H_0 \mathcal{U}') + \mathcal{U}'^\dagger H_0
  \mathcal{U}' + \mathcal{U}^\dagger(\mathcal{H'}\mathcal{U}) \nonumber
  \\
  &= H_0 + \mathcal{U}'^\dagger H_0 + \mathcal{U}'H_0 - \mathcal{X} + \mathcal{U}'^\dagger (\mathcal{U}' H_0 - \mathcal{X}) + \mathcal{U}^\dagger\mathcal{H'}\mathcal{U} \nonumber \\
  &= H_0 + (\mathcal{U}'^\dagger + \mathcal{U}' + \mathcal{U}'^\dagger \mathcal{U}')H_0 - \mathcal{X} - \mathcal{U}'^\dagger \mathcal{X} + \mathcal{U}^\dagger\mathcal{H'}\mathcal{U} \nonumber \\
  &= H_0 - \mathcal{X} - \mathcal{U}'^\dagger \mathcal{X} + \mathcal{U}^\dagger\mathcal{H'}\mathcal{U},
\end{align}
%
where the terms multiplied by $H_0$ cancel by unitarity.

The transformed Hamiltonian does not contain products by $H_0$ anymore, but it
does depend on $\mathcal{X}$, an auxiliary operator whose recurrent definition
we do not know yet.
To find it, we first focus on its anti-Hermitian part, $\mathcal{Z}$.
Since recurrence relations are expressions whose right hand side contains
Cauchy products between series, we need to find a way to make a product appear.
We do so by using the unitarity condition $\mathcal{U}'^\dagger + \mathcal{U}' =
-\mathcal{U}'^\dagger \mathcal{U}'$ to rewrite $\mathcal{Z}$:
%
\begin{align}
\label{eq:Z}
\mathcal{Z}
&= \frac{1}{2} (\mathcal{X} - \mathcal{X}^{\dagger}) \nonumber \\
&= \frac{1}{2}\Big[ (\mathcal{U}' + \mathcal{U}'^{\dagger}) H_0 - H_0 (\mathcal{U}' + \mathcal{U}'^{\dagger}) \Big] \nonumber \\
&= \frac{1}{2} \Big[ - \mathcal{U}'^{\dagger} (\mathcal{U}'H_0 - H_0 \mathcal{U}') + (\mathcal{U}'H_0 - H_0 \mathcal{U}')^{\dagger} \mathcal{U}' \Big] \nonumber \\
&= \frac{1}{2} (-\mathcal{U}'^{\dagger} \mathcal{X} + \mathcal{X}^{\dagger} \mathcal{U}').
\end{align}
%
Similar to computing $W_{\mathbf{n}}$, computing $Z_{\mathbf{n}}$ requires lower orders of
$\mathcal{X}$ and $\mathcal{U}'$, all blocks included.
This defines a recursive relation for $\mathcal{Z}$.
Then, we compute the Hermitian part of $\mathcal{X}$ by requiring that
$\tilde{\mathcal{H}}^{AB} = 0$ and find
%
\begin{align}
\label{eq:Y}
\mathcal{X}^{AB} = (\mathcal{U}^\dagger \mathcal{H}' \mathcal{U} -
\mathcal{U}'^\dagger \mathcal{X})^{AB}.
\end{align}
%
Once again, despite $\mathcal{X}$ enters the right hand side, because all the
terms lack \nth{0} order, this defines a recursive relation for $\mathcal{X}^{AB}$,
and therefore $\mathcal{Y}$.

The final part is standard: the definition of $\mathcal{Y}$ in \eqref{eq:XYZ} fixes
$\mathcal{V}$ as a solution of:
%
\begin{align}
\label{eq:sylvester}
\mathcal{V}^{AB}H_0^{BB} - H_0^{AA} \mathcal{V}^{AB} = \mathcal{Y}^{AB},
\end{align}
%
a Sylvester's equation, which we only need to solve once for every new order.
In the eigenbasis of $H_0$, the solution of Sylvester's equation is
$V^{AB}_{\mathbf{n}, ij} = Y^{AB}_{\mathbf{n}, ij}/(E_i - E_j)$, where $E_i$ are the eigenvalues of
$H_0$.
However, even if the eigenbasis of $H_0$ is not available, there are efficient
algorithms to solve Sylvester's equation, see below.

\subsection{Algorithm}

We now have the complete algorithm:
%
\begin{enumerate}
    \item Define series $\mathcal{U}'$ and $\mathcal{X}$ and make use of their block structure and Hermiticity.
    \item To define the diagonal blocks of $\mathcal{U}'$, use $\mathcal{W} = -\mathcal{U}'^\dagger\mathcal{U}'/2$.
    \item To find the off-diagonal blocks of $\mathcal{U}'$, solve Sylvester's equation $\mathcal{V}^{AB}H_0^{BB} - H_0^{AA}\mathcal{V}^{AB} = \mathcal{Y}^{AB}$.
      This requires $\mathcal{X}$.
    \item To find the diagonal blocks of $\mathcal{X}$, define $\mathcal{Z} = (-\mathcal{U}'^\dagger\mathcal{X} + \mathcal{X}^\dagger\mathcal{U}')/2$.
    \item For the off-diagonal blocks of $\mathcal{X}$, use $\mathcal{Y}^{AB} =
    (-\mathcal{U}'^\dagger\mathcal{X} +
     \mathcal{U}^\dagger\mathcal{H}'\mathcal{U})^{AB}$.
    \item  Compute the effective Hamiltonian as $\tilde{\mathcal{H}}_{\textrm{diag}} = H_0 - \mathcal{X} - \mathcal{U}'^\dagger \mathcal{X} + \mathcal{U}^\dagger\mathcal{H'}\mathcal{U}$.
\end{enumerate}

\subsection{Extra optimization: common subexpression elimination}
%
We further optimize the algorithm by reusing products that are needed in several
places.
%
Firstly, we rewrite the expressions for $\mathcal{Z}$ and $\tilde{\mathcal{H}}$
by utilizing the Hermitian conjugate of $\mathcal{U}'^\dagger \mathcal{X}$ without recomputing it:
%
\begin{gather*}
\mathcal{Z} = \frac{1}{2}[(-\mathcal{U}'^\dagger \mathcal{X})- \textrm{h.c.}],\\
\tilde{\mathcal{H}} = H_0 + \mathcal{U}^\dagger \mathcal{H}' \mathcal{U} - (\mathcal{U}'^\dagger \mathcal{X} + \textrm{h.c.}),
\end{gather*}
%
where $\textrm{h.c.}$ is the Hermitian conjugate, and $\mathcal{X}$ drops out from the diagonal blocks of $\tilde{\mathcal{H}}$ because diagonal of $\mathcal{X}$ is anti-Hermitian.
%
To compute $\mathcal{U}^\dagger \mathcal{H}' \mathcal{U}$ faster, we express it
using $\mathcal{F} \equiv \mathcal{H}'\mathcal{U}'$:
%
$$
\mathcal{U}^\dagger \mathcal{H}' \mathcal{U} = \mathcal{H}' + \mathcal{F} + \mathcal{F}^\dagger + \mathcal{U}'^\dagger \mathcal{F}.
$$
%
To further optimize the computations, we observe that some products appear both in $\mathcal{U}'^\dagger \mathcal{X}$ and $\mathcal{U}^\dagger \mathcal{H}' \mathcal{U}$.
%
To reuse these products, we separate the perturbation into diagonal and off-diagonal parts $\mathcal{H}' = \mathcal{H}'_\textrm{diag} + \mathcal{H}'_\textrm{offdiag}$.
%
We then introduce variables
%
\begin{align}
\mathcal{A} = \mathcal{H}'_\textrm{diag} \mathcal{U}', \quad
\mathcal{B} = \mathcal{H}'_\textrm{offdiag} \mathcal{U}', \quad
\mathcal{C} = \mathcal{X} - \mathcal{H}'_\textrm{offdiag}
\end{align}
%
This gives an updated expression for $\mathcal{Z}$:
%
\begin{align}
\label{eq:Z_optimized}
\mathcal{Z} = \frac{1}{2}(\mathcal{B}^\dagger - \mathcal{U}^\dagger\mathcal{C}) - \textrm{h.c.},
\end{align}
%
and more importantly for $\tilde{\mathcal{H}}$:
%
\begin{align}
\label{eq:H_tilde_optimized}
\tilde{\mathcal{H}} = H_0 + \mathcal{A} + \mathcal{A}^\dagger + (\mathcal{B} + \mathcal{B}^\dagger)/2 + \mathcal{U}'^\dagger (\mathcal{A} + \mathcal{B}) - (\mathcal{U}^\dagger \mathcal{C} + \textrm{h.c.})/2.
\end{align}

\subsection{Implicit method for large Hamiltonians}

Solving Sylvester's equation and computing the matrix products are the most
expensive steps of the algorithms for large Hamiltonians.
Pymablock can efficiently construct an effective Hamiltonian of a small
subspace even when the full Hamiltonian is a sparse matrix that is too costly to
diagonalize.
It does so by avoiding explicit computation of operators in $B$ subspace, and by
utilizing the sparsity of the Hamiltonian to compute the Green's function.
To do so, Pymablock uses either the MUMPS sparse solver via the python-mumps
wrapper or the KPM method, an approach originally introduced in
\cite{Irfan_2019}.

To implement this, we use the matrix $\Psi_A$ of the eigenvectors of the $A$
subspace to rewrite the Hamiltonian as
%
\begin{align}
\mathcal{H} \to \begin{pmatrix}
\Psi_A^\dagger \mathcal{H} \Psi_A & \Psi_A^\dagger \mathcal{H} P_B \\
P_B \mathcal{H} \Psi_A & P_B \mathcal{H} P_B
\end{pmatrix},
\end{align}
%
where $P_B = 1 - \Psi_A \Psi_A^\dagger$ is the projector onto the $B$ subspace.
This Hamiltonian is larger in size than the original one because the $B$ block has
additional null vectors corresponding to the $A$ subspace.
This, however, allows to preserve the sparsity structure of the Hamiltonian by applying
$P_B$ and $\mathcal{H}$ separately.
Additionally, applying $P_B$ is efficient because $\Psi_A$ is a low rank matrix.
We then perform perturbation theory of the rewritten $\mathcal{H}$.

To solve the Sylvester's equation for the modified Hamiltonian, we write it for
every row of $V_n^{AB}$ separately:
%
\begin{align}
V_{n, ij}^{AB} (E_i - H_0) = Y_{n, j}
\end{align}
%
This equation is well-defined despite $E_i - H_0$ is not invertible because
$Y_{n}$ has no components in the $A$ subspace.

\section{Implementation}

\co{To implement the algorithms, we need a data structure that represents a
multidimensional series of block matrices.}
To implement the algorithm, we need a data structure that represents a
multidimensional series of operators, where dimensions label independent
perturbations.
Additionally, the data structure needs to label blocks, so that the algorithm
supports several forms of input, e.g. dense arrays, sparse matrices, symbolic
expressions, an implicit subspace, or a custom Python object that supports
products and sums.
Manipulating blocks also allows to compute the effective Hamiltonian without
explicitly constructing the full Hamiltonian, which is useful for Hamiltonians
with a large $BB$ subspace that is costly to store and compute.
To run the recursion, the series needs to be queryable by order and block.
This is also useful in cases where the user may want terms that combine
different perturbations, or when the user wants to compute more terms than
originally requested.
Lastly, the data structure needs to support a block-wise multivariate Cauchy
product, which is the main operation in the recursion and is used to compute
the transformed Hamiltonian.

\co{We address this by defining a BlockSeries class.}
To address these requirements, we define a \mintinline{python}|BlockSeries|
Python class and use it to represent the series of $\mathcal{U}$,
$\mathcal{H}$, and $\tilde{\mathcal{H}}$.
A \mintinline{python}|BlockSeries| is a Python object equipped with a function
to compute its elements and a dictionary to cache the results.
For example, the \mintinline{python}|BlockSeries| for $\tilde{\mathcal{H}}$ has
a function that computes the block-wise multivariate Cauchy product in
Eq.~\eqref{eq:H_tilde_optimized}, \mintinline{python}|compute_H_tilde|.
%
\begin{minted}{python}
    H_tilde = BlockSeries(
        shape=(2, 2), # 2 blocks
        n_infinite=1, # number of perturbative parameters
        eval=compute_H_tilde,
    )
\end{minted}
%
To get the elements of the series, we implement Numpy array indexing,
which allows us to request several elements at once by using tuples and slices.
\mintinline{python}|H_tilde[0, 0, 2]| returns the $AA$ block of
$\tilde{\mathcal{H}}$ at order 2 in the perturbation, and
\mintinline{python}|H_tilde[0, 0, :3]| returns the $AA$ block of all orders up
to 2 in a Numpy masked array.
In the latter, the masked array only contains non-zero elements, a feature that
we use each time we compute the Cauchy products between series, avoiding
unnecessary operations.

\co{Using the BlockSeries interface allows us to implement a range of
optimizations that go beyond directly implementing the polynomial
parametrization}
Not only does the \mintinline{python}|BlockSeries| interface allow us to
implement the polynomial parametrization of the unitary transformation, but
also several other optimizations.
For example, we exploit Hermiticity when computing the Cauchy product of the
diagonal blocks of $\mathcal{U}$ and $\tilde{\mathcal{H}}$, because we only
compute half of the matrix products and then complex conjugate the result to
obtain the rest.
Similarly, we avoid computing $BA$ blocks of $\mathcal{U}$ and
$\tilde{\mathcal{H}}$ by providing a function to the \mintinline{python}|BlockSeries| that returns
the conjugate transpose of the respective $AB$ blocks.
As a result, whenever we query a $BA$ block, we first compute the $AB$ block,
store it, and then compute the $BA$ block directly.
This procedure brings an additional advantage: because the terms that compose
$AB$ blocks contain matrix products that first multiply small matrices and then
the large ones, it saves computational time and memory.
For example, the term $V_{n -i}^{AB} H_0^{BB}
W_i^{BB}$ in $Y_n$ is systematically computed as $(V_{n -i}^{AB}
H_0^{BB}) W_i^{BB}$ instead of $V_{n -i}^{AB}
(H_0^{BB} W_i^{BB})$.
This is only one example of how the \mintinline{python}|BlockSeries| interface allows us to
implement a symmetrized algorithm, and we leave other symmetries for future
work.
Such an extension would be useful for systems where $\mathcal{U}$ or
$\tilde{\mathcal{H}}$ vanish due to symmetries, so that the zero blocks can be
skipped beforehand.
\todo{Remove this paragraph?}

\co{To deal with an implicit B subspace, we use MUMPS and LinearOperators.}
Because \mintinline{python}|BlockSeries| can represent any input type, we use
it to directly implement the implicit algorithm for large sparse Hamiltonians.
To do this, we use the matrix $\Psi_A$ of the eigenvectors of the $A$ subspace
to rewrite the Hamiltonian as
%
\begin{align}
\mathcal{H} \to \begin{pmatrix}
\Psi_A^\dagger \mathcal{H} \Psi_A & \Psi_A^\dagger \mathcal{H} P_B \\
P_B \mathcal{H} \Psi_A & P_B \mathcal{H} P_B
\end{pmatrix},
\end{align}
%
where $P_B = 1 - \Psi_A \Psi_A^\dagger$ is the projector onto the $B$ subspace.
This Hamiltonian is larger in size than the original one because the $B$ block
has additional null vectors corresponding to the $A$ subspace.
This, however, allows to preserve the sparsity structure of the Hamiltonian by
applying $P_B$ and $\mathcal{H}$ separately.
Additionally, applying $P_B$ is efficient because $\Psi_A$ is a low rank matrix.
We then perform perturbation theory of the rewritten $\mathcal{H}$.
To solve the Sylvester's equation for the modified Hamiltonian, we write it for
every row of $V_n^{AB}$ separately:
%
\begin{align}
V_{n, ij}^{AB} (E_i - H_0) = Y_{n, j}
\end{align}
%
This equation is well-defined despite $E_i - H_0$ is not invertible because
$Y_{n}$ has no components in the $A$ subspace.
Here we also wrap the projector $P_B$ by a \mintinline{python}{LinearOperator}
object from \mintinline{python}{Scipy}.
This allows us to compute matrix-vector products between $P_B$ and a vector,
without explicitly constructing $P_B$ or any other product between elements of
the $B$ subspace, keeping the memory usage low.
For the same purpose, we use the MUMPS sparse solver
\cite{Amestoy_2001},
\cite{Amestoy_2006}, or the KPM method
\cite{Wei_e_2006}, to compute the Green's function of the
$B$ subspace.
As a consequence, the implicit algorithm can be used on matrices with millions
of degrees of freedom as long as they are sparse.

\co{Finally, we implement an overall function that interprets the user inputs and
returns a BlockSeries for the transformed Hamiltonian.}
On the other hand, the standard algorithm explicitly manipulates both subspaces,
but can work with dense matrices too.
We use the eigenvectors of the $A$ and $B$ subspaces to project the input
Hamiltonian and represent it with a \mintinline{python}|BlockSeries|, a
procedure that works for numerical and symbolic matrices.
Because we aim for an easy-to-use interface, we implement a function that
interprets the user inputs and decides which algorithm to use: the implicit
method if only the $A$ subspace is provided, and the standard algorithm
otherwise.
This is \mintinline{python}|block_diagonalize|, the only function that the user
needs to call.
If $H_0$ is diagonal and a custom function to solve Sylvester's equation is not
provided to \mintinline{python}|block_diagonalize|, Pymablock uses a default
function to compute the energy denominators.

\section{Benchmark}
\label{sec:benchmark}

\subsection{Comparison to other methods}

\co{Pymablock is not only efficient, but its implementation has potential
to be expanded to other settings, like time-dependent Hamiltonians, many-body
Hamiltonians, and continuum Hamiltonians.}
Because Pymablock supports a wide range of inputs and custom solvers for
Sylvester's equation, its application is not limited to the examples shown
here.
Pymablock can be used to find effective Hamiltonians for interacting systems,
infinitely-sized Hilbert spaces, and Hamiltonians with continuum degrees of
freedom, by providing Hamiltonians written in second quantization form.
This is advantageous over other methods, which are limited to pre-defining
an appropriate generator for the unitary transformation.
Similarly, this flexibility allows Pymablock to work with time-dependent
Hamiltonians, an extension that we leave for future work.


\co{To demonstrate the efficiency of the implicit algorithm, we show its
time scaling compared to sparse diagonalizaton.}


The effective Hamiltonian is constructed for the $10$ lowest energy
states of a $52 \times 52$ 2D square lattice with a nearest-neighbor
hopping and random on-site potential.
\begin{figure}[h]
    \centering
    \includegraphics[width=\textwidth]{figures/benchmark_bandstructure.pdf}
    \caption{
        Bandstructure of an effective Hamiltonian (black) compared to exact
        sparse diagonalization (gray).
        The time spent in sparse diagonalization for one value of $\delta \mu$
        (gray) is shown in the lowest panel.
        This is also a constant overhead for the construction of the effective
        Hamiltonian.
        The time spent in the LU decomposition of the Hamiltonian (red) is a
        one-time cost, together with the time spent into getting second
        and third orders.
        First and second order corrections are negligible, not shown.
        }
    \label{fig:benchmark_bandstructure}
\end{figure}

\co{}

\begin{figure}[h]
    \centering
    \includegraphics[width=\textwidth]{figures/benchmark_matrix_products.pdf}
    \caption{
        (left) Matrix products scaling per order of $\tilde{\mathcal{H}}$.
        (right) ???
    }
    \label{fig:benchmark_matrix_products}
\end{figure}
\subsection{Time scaling}


\subsection{Error scaling}

Show error accumulation, show that the inverse of the transformation holds to numerical precision.

\section{Conclusion}

\co{Pymablock's algorithm combines advantages of other perturbation theory methods.}
We developed an algorithm for constructing an effective Hamiltonian that combines advantages of different perturbative expansions.
The main building block of our approach is a set of recurrence relations that define several series that depend on each other and combine into the effective Hamiltonian.
Our algorithm constructs the same effective Hamiltonians as the Schrieffer--Wolff transformation~\cite{Schrieffer_1966} in the case of $2$ subspaces, while keeping the linear scaling per extra order similar to the density matrix perturbation theory~\cite{McWeeny_1962, Truflandier_2020} or the non-orthogonal perturbation theory~\cite{Bloch_1958}.
Its expressions minimize the number of matrix multiplications per order, making it appealing both for symbolic and numerical computations.
Pymablock's algorithm is more general than any other perturbation theory method we are aware of, because it performs multi-block diagonalization and selective diagonalization with a single algorithm.

\co{The package provides a universal interface that handles constructing effective models in all quantum mechanical systems.}
We provide a Python implementation of the algorithm in the Pymablock package~\cite{Araya_2024}.
The package is thoroughly tested (95\% test coverage as of version 2.1), becoming a reliable tool for constructing effective Hamiltonians that combine multiple perturbations to high orders.
The core of the Pymablock interface is the \mintinline{python}|BlockSeries| class that handles arbitrary objects as long as they support algebraic operations.
This enables Pymablock's construction of effective models for large tight-binding models using its implicit method as well as for second quantized Hamiltonians upon providing a custom Sylvester equation solver.
It also allows Pymablock to solve both symbolic and numerical problems in diverse physical settings, and potentially to incorporate it into existing packages, such as scqubits~\cite{Groszkowski_2021}, QuTiP~\cite{Johansson_2012,Johansson_2013}, or dft2kp~\cite{Cassiano_2024}.

\co{The package provides a foundation to implement other perturbative expansions.}
Beyond the Schrieffer--Wolff transformation, the Pymablock package provides a foundation for defining other perturbative expansions.
We anticipate extending it to time-dependent problems, where the different regimes of the time-dependent drive modify the recurrence relations that need to be solved~\cite{Rodriguez-Vega_2018,Malekakhlagh_2020}.
Applying the same framework to problems with weak position dependence would allow to construct a nonlinear response theory of quantum materials.
These two extensions are active areas of research~\cite{Motzoi_2009,Bernevig_2021,Theis_2018,Venkatraman_2022,Xu_2024b, Reascos_2024}.
Finally, we expect that in the many-particle context the same framework supports implementing different flavors of diagrammatic expansions.

\section*{Acknowledgements}
We thank Valla~Fatemi and Antonio~Manesco for feedback on the manuscript.

\section*{Data availability}
The code used to produce the reported results is available on Zenodo~\cite{Araya_2024}.

\paragraph{Author contributions}
A.~R.~A. had the initial idea and oversaw the project.
All authors developed the algorithm.
I.~A.~D., S.~M., H.~K.~K, and A.~R.~A. wrote the package.
I.~A.~D. and A.~R.~A. wrote the paper.

\paragraph{Funding information}
This research was supported by the Netherlands Organization for Scientific Research (NWO/OCW) as part of the Frontiers of Nanoscience program, a NWO VIDI grant 016.Vidi.189.180, and OCENW.GROOT.2019.004.
D.V. acknowledges funding from the Deutsche Forschungsgemeinschaft (DFG, German Research Foundation) under Germany’s Excellence Strategy through the W\"{u}rzburg-Dresden Cluster of Excellence on Complexity and
Topology in Quantum Matter – ct.qmat (EXC 2147, project-ids 390858490 and 392019).


%%%%%%%%%%%%%%%%%%%%%%%%%%%%%%%%%%%%%%%%%%%%%%%%%%
%%%%%%%%%%%%%%  acronyms & glossary  %%%%%%%%%%%%%
\printglossaries
%%%%%%%%%%%%%%%%%%%%%%%%%%%%%%%%%%%%%%%%%%%%%%%%%%


\bibliography{main.bib}

\nolinenumbers

\end{document}
