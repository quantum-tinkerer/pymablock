\section{Implementation}

\co{To implement the algorithms, we need a data structure that represents a
multidimensional series of block matrices.}
To implement the algorithms, we need a data structure that represents a
multidimensional series of operators, where dimensions label independent
perturbations.
Additionally, the data structure needs to label blocks, so that the algorithm
supports several forms of input, e.g. dense arrays, sparse matrices, symbolic
expressions, an implicit subspace, or a custom Python object.
Manipulating blocks also allows to compute the effective Hamiltonian without
explicitly constructing the full Hamiltonian, which is useful for Hamiltonians
with a large $BB$ subspace that is costly to store and compute.
To run the recursion, the series needs to be queryable by order and block.
This is useful in cases where the user may want terms that combine different
perturbations, or when the user wants to compute more terms than originally
requested.
Lastly, the data structure needs to support a block-wise multivariate Cauchy
product, which is the main operation in the recursion and is used to compute
the transformed Hamiltonian.

\co{We address this by defining a BlockSeries class.}
To address these requirements, we define a \mintinline{python}{BlockSeries} Python class and use
it to represent the series of $\mathcal{U}$, $\mathcal{H}$, and
$\tilde{\mathcal{H}}$.
A \mintinline{python}{BlockSeries} is a Python object equipped with a function to compute its
elements and a dictionary to cache the results.
For example, the \mintinline{python}{BlockSeries} for $\tilde{\mathcal{H}}$ has a function that
computes the block-wise multivariate Cauchy product of $\mathcal{U}^{\dagger}
\mathcal{H} \mathcal{U}$.
To get the elements of the series, we implement Numpy array indexing,
which allows us to request several elements at once by using tuples and slices.
In this case, the \mintinline{python}{BlockSeries} returns a masked array that only contains
non-zero elements, a feature that we use each time we compute the Cauchy
products between series, avoiding unnecessary operations.

\co{Using the BlockSeries interface allows us to implement a range of
optimizations that go beyond directly implementing the polynomial
parametrization}
Not only does the \mintinline{python}{BlockSeries} interface allow us to implement the polynomial
parametrization of the unitary transformation, but also several other
optimizations.
For example, we exploit Hermiticity when computing the Cauchy product of the
diagonal blocks of $\mathcal{U}$ and $\tilde{\mathcal{H}}$, because we only
compute half of the matrix products and then complex conjugate the result to
obtain the rest.
Similarly, we avoid computing $BA$ blocks of $\mathcal{U}$ and
$\tilde{\mathcal{H}}$ by providing a function to the \mintinline{python}{BlockSeries} that returns
the conjugate transpose of the respective $AB$ blocks.
As a result, whenever we query a $BA$ block, we first compute the $AB$ block,
store it, and then compute the $BA$ block directly.
This procedure brings an additional advantage: because the terms that compose
$AB$ blocks contain matrix products that first multiply small matrices and then
the large ones, it saves computational time and memory.
For example, the term $V_{n -i}^{AB} H_0^{BB}
W_i^{BB}$ in $Y_n$ is systematically computed as $(V_{n -i}^{AB}
H_0^{BB}) W_i^{BB}$ instead of $V_{n -i}^{AB}
(H_0^{BB} W_i^{BB})$.
This is only one example of how the \mintinline{python}{BlockSeries} interface allows us to
implement a symmetrized algorithm, and we leave other symmetries for future
work.
Such an extension would be useful for systems where $\mathcal{U}$ or
$\tilde{\mathcal{H}}$ vanish due to symmetries, so that the zero blocks can be
skipped beforehand.

\co{To deal with an implicit B subspace, we use MUMPS and LinearOperators.}
Because \mintinline{python}{BlockSeries} can represent any input type, we use it to directly
implement the \mintinline{python}{implicit} algorithm.
To do so, we construct a \mintinline{python}{BlockSeries} with the block structure in Equation
(\ref{H_implicit}), and wrap the projector $P_B$ by a \mintinline{python}{LinearOperator} object
from \mintinline{python}{Scipy}.
This allows us to compute matrix-vector products between $P_B$ and a vector,
without explicitly constructing $P_B$ or any other product between elements of
the $B$ subspace, keeping the memory usage low.
For the same purpose, we use the MUMPS sparse solver
\cite{Amestoy_2001},
\cite{Amestoy_2006}, or the KPM method
\cite{Wei_e_2006}, to compute the Green's function of the
$B$ subspace.
As a consequence, the \mintinline{python}{implicit} algorithm can be used on matrices with
millions of degrees of freedom as long as they are sparse.

\co{Finally, we implement an overall function that interprets the user inputs and
returns a BlockSeries for the transformed Hamiltonian.}
On the other hand, the \mintinline{python}{general} and \mintinline{python}{expanded} algorithms explicitly
manipulate both subspaces, but can work with dense matrices too.
We use the eigenvectors of the $A$ and $B$ subspaces to project the input
Hamiltonian and represent it with a \mintinline{python}{BlockSeries}, a procedure that works for
numerical and symbolic matrices.
Because we aim for an easy-to-use interface, we implement a function that
interprets the user inputs and decides which algorithm to use: \mintinline{python}{expanded}
for symbolic matrices, \mintinline{python}{general} for numerical matrices, and \mintinline{python}{implicit} if
only the $A$ subspace is provided.
This is \mintinline{python}{block\_diagonalize}, the only function that the user needs to call.
By default, \mintinline{python}{block\_diagonalize} uses a default function to solve Sylvester's
equation if $H_0$ is diagonal, but the user can provide a custom one instead.
