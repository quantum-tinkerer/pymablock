\section{Introduction}

\co{Effective models enable the study of complex physical systems by reducing the space of interest to a low-energy one.}
Effective models enable the study of complex quantum systems by reducing the dimensionality of the Hilbert space.
Their construction separates the low and high-energy subspaces by block-diagonalizing a perturbed Hamiltonian
%
\begin{equation}
    \mathcal{H} = \begin{pmatrix}H_0^{AA} & 0 \\ 0 & H_0^{BB}\end{pmatrix} + \mathcal{H}',
\end{equation}
%
where $H_0^{AA}$ and $H_0^{BB}$ are separated by an energy gap, and $\mathcal{H}'$ is a series in a perturbative parameter.
This procedure requires finding a series of the basis transformation $\mathcal{U}$ that is unitary and that also cancels the off-diagonal block of the transformed Hamiltonian order by order, as shown in Fig.~\ref{fig:block_diagonalization}.
The low-energy effective Hamiltonian $\tilde{\mathcal{H}}^{AA}$ is then a series in the perturbative parameter, whose eigenvalues and eigenvectors are approximate solutions of the complete Hamiltonian.
As a consequence, the effective model is sufficient to describe the low-energy properties of the original system while also being simpler and easier to handle.

\co{A standard approach to constructing the effective model is the Schrieffer-Wolff algorithm.}
A common approach to constructing an effective Hamiltonian is the Schrieffer--Wolff transformation~\cite{Schrieffer_1966,Bravyi_2011}, also known as Löwdin partitioning~\cite{Lowdin_1962}, or quasi-degenerate perturbation theory.
This method parameterizes the unitary transformation $\mathcal{U} = e^{-\mathcal{S}}$ and finds the series $\mathcal{S}$ that decouples the $A$ and $B$ subspaces of $\tilde{\mathcal{H}} = e^{\mathcal{S}}\mathcal{H}e^{-\mathcal{S}}$.
This idea enabled advances in multiple fields of quantum physics.
As an example, all the k.p models are a result of treating crystalline momentum as a perturbation that only weakly mixes atomic orbitals separated in energy~\cite{Luttinger_1955,Winkler_2003,McCann_2013,Bernevig_2021}.
More broadly, this method serves as a go-to tool in the study of superconducting circuits and quantum dots, where couplings between circuit elements and drives are treated as perturbations to reproduce the dynamics of the system~\cite{Krantz_2019,Romhanyi_2015}.
Applied to time-dependent Hamiltonians, the Schrieffer--Wolff transformation is an essential tool for the design of quantum gates~\cite{Malekakhlagh_2020, Petrescu_2023}.
%
\begin{figure}[h!]
    \centering
    \includegraphics[width=\textwidth]{figures/diagrams_H.pdf}
    \caption{
      Block-diagonalization of a Hamiltonian with a first order perturbation.
    }
    \label{fig:block_diagonalization}
\end{figure}

\co{Even though these methods are standard, their algorithm is computationally expensive, scaling poorly for large systems and high orders.}
Constructing effective Hamiltonians is, however, both algorithmically complex and computationally expensive.
This is a consequence of the recursive equations that define the unitary transformation, which require an exponentially growing number of matrix products in each order.
In particular, already a 4-th order perturbative expansion that is necessary for many applications may require hundreds of terms.
While the computational complexity is only a nuisance when analysing model systems, it becomes a bottleneck whenever the Hilbert space is high-dimensional.
Several other approaches improve the performance of the Schrieffer--Wolff algorithm by either using different parametrizations of the unitary transformation~\cite{Van_Vleck_1929, Lowdin_1962, Shavitt_1980, Klein_1974, Suzuki_1983}, adjusting the problem setting to density matrix perturbation theory~\cite{McWeeny_1962, Truflandier_2020}, or a finding a similarity transform instead of a unitary~\cite{Bloch_1958}.
An alternative formulation of the perturbative diagonalization uses Wegner's flow equation~\cite{Wegner_1994,Kehrein_2007} to construct a continuous unitary transformation (CUT) that depends on a fictitious flow parameter, which at infinity eliminates the undesired terms from the Hamiltonian~\cite{Knetter_2000,Oitmaa_2006}.
CUT is common in the study of many-body systems~\cite{Krull_2012}, and it relies on solving a set of differential equations to obtain the effective Hamiltonian.
A more recent line of research even applies the ideas of Schrieffer--Wolff transformation to quantum algorithms for the study of many-body systems~\cite{Wurtz_2020, Zhang_2022}.
Despite these advances, neither of the approaches combines an optimal scaling with the ability to construct effective Hamiltonians.

\co{Existing algorithms do not generalize beyond two subspaces.}
Another limitation of the Schrieffer--Wolff transformation is that it only decouples two subspaces at a time.
While a straightforward generalization of the Schrieffer--Wolff transformation to multiple subspaces is to decouple one block at a time, this approach is suboptimal and depends on the order in which the blocks are decoupled.
The literature on multi-block diagonalization is scarce and considers two approaches: the least action or the block-diagonality of the generator~\cite{Mankodi_2024}.
The former constructs a unitary transformation that is as close as possible to the identity, and the latter constructs a block off-diagonal unitary similar to the Schrieffer--Wolff generator.
These approaches are useful to design gates for superconducting qubits~\cite{Magesan_2020} and to characterize nonlocal interactions in multi-qubit systems~\cite{Xu_2024a}, both of which require the decoupling of qubit subspaces from different sets of higher energy states.
Reference~\cite{Mankodi_2024}, however, showed that the two generalizations of the Schrieffer--Wolff transformation yield different effective Hamiltonians when applied to more than two subspaces.
While the perturbative CUT method naturally decouples multiple subspaces~\cite{Knetter_2003}, in general solving the differential equations inherent to the method may become a computational bottleneck.
To our knowledge, there is no general algorithm that constructs effective Hamiltonians for multiple subspaces directly from the least action principle, and how to do so is an open question.

\co{We develop an efficient algorithm capable of symbolic and numeric computations and make it available in Pymablock.}
We introduce an algorithm to construct effective models with optimal scaling, thus making it possible to find high order effective models for systems with millions of degrees of freedom.
This algorithm exploits the efficiency of recursive evaluations of series satisfying polynomial constraints and obtains the same effective Hamiltonian as the Schrieffer--Wolff transformation in the case of two subspaces.
Pymablock's algorithm, however, deals with any number of subspaces, providing a generalization of the Schrieffer--Wolff transformation for multi-block diagonalization and selective decoupling between any two states.
We make the algorithm available via the open source package Pymablock \footnote{The documentation and tutorials are available in \url{https://pymablock.readthedocs.io/}}(PYthon MAtrix BLOCK-diagonalization), a versatile tool for the study of numerical and symbolic models.
