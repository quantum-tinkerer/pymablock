\section{Benchmark}
\label{sec:benchmark}

\co{Pymablock is more efficient than a direct implementation of a Schrieffer--Wolff transformation.}
To the best of our knowledge, there are no other packages implementing arbitrary order quasi-degenerate perturbation theory.
Literature references provide explicit expressions for the effective Hamiltonian up to fourth order, together with the procedure for obtaining higher order expressions~\cite{Winkler_2003}.
Because the full reference expressions are lengthy, we do not provide them, but for example at $4$-th order the effective Hamiltonian is a sum of several expressions of the form
\begin{equation}
\label{eq:SW_term}
\frac{H'_{mm^{''}}H'_{m^{''}m^{'''}}H'_{m^{'''}l}H'_{lm^{'}}}{(E_{m^{''}}-E_{l})(E_{m^{'''}}-E_{l})(E_{m}-E_{l})}.
\end{equation}
More generally, at $n$-th order each term has a product of $n$ matrix elements of the Hamiltonian and $n-1$ energy denominators.
Directly carrying out the summation over all the states requires $\mathcal{O}((N_A + N_B)^{n+1})$ operations, where $N_A$ and $N_B$ are the number of states in the two subspaces.
In other words, it scales with the problem size worse than a matrix product.
Formulating Eq.~\eqref{eq:SW_term} as $n-1$ matrix products combined with $n-1$ solutions
of Sylvester's equation, brings this complexity down to $\mathcal{O}((n-1) \times (N_A + N_B)^3)$.
This allows us to evaluate the reference expressions for the effective Hamiltonian for
$0$-th, $1$-st, $2$-nd, $3$-rd, and $4$-th order using $0$, $0$, $1$, $4$, and $27$ matrix products, respectively.
With Pymablock, we need $0$, $0$, $1$, $7$, and $24$, matrix products to obtain the same orders of the effective Hamiltonian.
However, because we the \mintinline{Python}|BlockSeries| class masks out zero elements,
the number of matrix products depends on the sparsity structure of the Hamiltonian, as shown in Figure \ref{fig:benchmark_matrix_products}.
%
\begin{figure}[h]
    \centering
    \includegraphics[width=\textwidth]{figures/benchmark_matrix_products.pdf}
    \caption{
        Matrix products computed per $\tilde{\mathcal{H}}^{AA}_{n}$ for
        a dense and block off-diagonal first-order perturbation (left) and a dense and block off-diagonal summation of perturbations to infinite order (right).
    }
    \label{fig:benchmark_matrix_products}
\end{figure}
%

\co{To demonstrate the efficiency of the implicit method, we show its time scaling compared to sparse diagonalizaton.}
We demonstrate the performance of the implicit method by using it to compute the low-energy spectrum of a large tight-binding model, and comparing its time consumption to the one of sparse diagonalization.
For the model, we choose a 2D square lattice
of $52 \times 52$ sites with nearest-neighbor hopping and a random on-site potential, for which we tune the parameters so that there is a band gap and avoided crossings, as shown in Figure \ref{fig:benchmark_bandstructure}.
Like in Sec.~\ref{sec:induced_gap}, constructing the effective Hamiltonian involves three steps.
First, we sparse diagonalize the unperturbed Hamiltonian to obtain the its $10$ lowest states and orthonormalize them.
Second, we call \mintinline{python}|block_diagonalize|, which applies an LU decomposition to the Hamiltonian.
Third, we compute corrections to $\tilde{H}$ for a fixed value of $\delta \mu$, which we then rescale to obtain the effective Hamiltonian at different values of $\delta \mu$.
Each of these steps is a one-time cost, independent of the sampled values of $\delta \mu$, see Figure \ref{fig:benchmark_bandstructure}.
Finally, we diagonalize each $10 \times 10$ effective Hamiltonian to obtain the low-energy spectrum, however, this is a negligible cost compared to the other steps.
As a consequence, if one is interested in the bandstructure for a range of $\delta \mu$, it is more efficient to apply the implicit method than to sparse diagonalize the Hamiltonian for each value of $\delta \mu$, even if these are just $2$ points to sample.
%
\begin{figure}[h]
    \centering
    \includegraphics[width=\textwidth]{figures/benchmark_bandstructure.pdf}
    \caption{
        Bandstructure of the effective Hamiltonian (black) of a
        tight-binding model compared to exact sparse diagonalization (gray).
        The time spent in sparse diagonalization for one value of $\delta \mu$ (gray) and $10$ lowest states is shown in the lowest panel (gray).
        This time is comparable to a one-time LU decomposition (red) of the Hamiltonian, plus first, second (yellow) and third (blue) order corrections to the effective Hamiltonian.
        }
    \label{fig:benchmark_bandstructure}
\end{figure}
