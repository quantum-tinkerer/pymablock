\section{Benchmark}
\label{sec:benchmark}

\co{Pymablock is more efficient than a direct implementation of a Schrieffer--Wolff transformation.}
To the best of our knowledge, there are no other packages implementing arbitrary order quasi-degenerate perturbation theory.
Literature references provide explicit expressions for the effective Hamiltonian up to fourth order, together with the procedure for obtaining higher order expressions~\cite{Winkler_2003}.
Because the full reference expressions are lengthy, we do not provide them, but for example at $4$-th order the effective Hamiltonian is a sum of several expressions of the form
\begin{equation}
\label{eq:SW_term}
\sum_{m^{''} m^{'''} l}
\frac{H'_{mm^{''}}H'_{m^{''}m^{'''}}H'_{m^{'''}l}H'_{lm^{'}}}{(E_{m^{''}}-E_{l})(E_{m^{'''}}-E_{l})(E_{m}-E_{l})},
\end{equation}
where the $m$-indices label states from the $A$-subspace and $l$-indices label the states from the $B$-subspace.
More generally, at $n$-th order each term is a product of $n$ matrix elements of the Hamiltonian and $n-1$ energy denominators.
Directly carrying out the summation over all the states requires $\mathcal{O}(N_A^2 N_B^{n-1})$ operations, where $N_A$ and $N_B$ are the number of states in the two subspaces.
In other words, the direct computation scales worse than a matrix product with the problem size.
Formulating Eq.~\eqref{eq:SW_term} as $n-1$ matrix products combined with $n-1$ solutions
of Sylvester's equation, brings this complexity down to $\mathcal{O}((n-1) \times N_A N_B^2)$.
This optimization, together with the hermiticity of the sum, allows us to evaluate the reference expressions for the effective Hamiltonian for $2$-nd, $3$-rd, and $4$-th order using $1$, $4$, and $27$ matrix products, respectively.
On the other hand, Pymablock's algorithm utilizes $1$, $7$, and $24$, matrix products to obtain the same orders of the effective Hamiltonian.
The additional products that Pymablock performs at third order are due to terms that cancel out in $\tilde{H}_{3}$, and we aim to optimize this in future versions of the package.
Starting from fourth order computations, the scaling of Pymablock's algorithm outperforms the exponential growth of the number of terms in the reference expressions, and enables computations at all realistically useful orders.
Moreover, because the \mintinline{Python}|BlockSeries| class tracks absent terms, in practice the number of matrix products depends on the sparsity of the block structure of the perturbation, as shown in Figure \ref{fig:benchmark_matrix_products}.
%
\begin{figure}[h]
    \centering
    \includegraphics[width=\textwidth]{figures/benchmark_matrix_products.pdf}
    \caption{
        Matrix products required to compute $\tilde{H}^{AA}_{n}$ for
        a dense and block off-diagonal first-order perturbation (left) and a dense and block off-diagonal perturbative series with terms of all orders present (right).
        }
    \label{fig:benchmark_matrix_products}
\end{figure}
%

\co{The entire implicit method costs less than a single sparse diagonalization.}
The efficiency of Pymablock becomes especially apparent when applied to sparse numerical problems, similar to Sec.~\ref{sec:induced_gap}.
We demonstrate the performance of the implicit method by using it to compute the low-energy spectrum of a large tight-binding model, and comparing Pymablock's time cost to that of sparse diagonalization.
We define a 2D square lattice of $52 \times 52$ sites with nearest-neighbor hopping and a random onsite potential $\mu(\mathbf{r})$.
The perturbation $\delta \mu (\mathbf{r})$ interpolates between two different disorder realizations.
For the sake of an illustration, we choose the system's parameters such that the dispersion of the lowest few levels with $\delta \mu$ features avoided crossings and an overall nonlinear shape, whose details are not relevant.
Similar to Sec.~\ref{sec:induced_gap}, constructing the effective Hamiltonian involves three steps.
First, we compute the $10$ lowest states of the unperturbed Hamiltonian using sparse diagonalization.
Second, \mintinline{python}|block_diagonalize| computes a sparse LU decomposition of the Hamiltonian at each of the $10$ eigenenergies.
Third, we compute corrections $\tilde{H}_1$, $\tilde{H}_2$, and $\tilde{H}_3$ to the effective Hamiltonian, each being a $10 \times 10$ matrix.
Each of these steps is a one-time cost, see Figure \ref{fig:benchmark_bandstructure}.
Finally, to compare the perturbative calculation to sparse diagonalization, we construct the effective Hamiltonian $\tilde{H} = H_0 + \delta \mu \tilde{H}_1 + \delta \mu^2 \tilde{H}_2 + \delta \mu^3 \tilde{H}_3$ and diagonalize it to obtain the low-energy spectrum for a range of $\delta \mu$.
This has a negligible cost compared to constructing the series.
The comparison is shown in Fig.~\ref{fig:benchmark_bandstructure}.
We observe that while the second order results are already very close to the exact spectrum, the third order corrections fully reproduce the sparse diagonalization.
At the same time, the entire cost of computing the perturbative band structure for a range of $\delta \mu$ is lower than computing a single additional sparse diagonalization.
%
\begin{figure}[h]
    \centering
    \includegraphics[width=\textwidth]{figures/benchmark_bandstructure.pdf}
    \caption{
        Top panels: band structure of the perturbative effective Hamiltonian (black) of a tight-binding model compared to exact sparse diagonalization (gray).
        Bottom panel: a comparison of the Pymablock's time cost with sparse diagonalization.
        Most of the time is spent in the LU decomposition of the Hamiltonian (red).
        The entire cost of the implicit method is lower than a single sparse diagonalization (gray).
        The operations of negligible cost are not shown.
        }
    \label{fig:benchmark_bandstructure}
\end{figure}
